
% Thesis Abstract -----------------------------------------------------


%\begin{abstractslong}    %uncommenting this line, gives a different abstract heading
\begin{abstracts}        %this creates the heading for the abstract page

The work in this thesis is concerned with the study of bacterial adaptation on short and long timescales. In the first section, consisting of three chapters, I describe a recently developed high-throughput technology for probing gene function, transposon-insertion sequencing, and its application to the study of functional differences between two important human pathogens, \textit{Salmonella enterica} subspecies \textit{enterica} serovars Typhi and Typhimurium. In a first study, I use transposon-insertion sequencing to probe differences in gene requirements during growth on rich laboratory media, revealing differences in serovar requirements for genes involved in iron-utilization and cell-surface structure biogenesis, as well as in requirements for non-coding RNA. In a second study I more directly probe the genomic features responsible for differences in serovar pathogenicity by analyzing transposon-insertion sequencing data produced following a two hour infection of human macrophage, revealing large differences in the selective pressures felt by these two closely related serovars in the same environment.

The second section, consisting of two chapters, uses statistical models of sequence variation, i.e. covariance models, to examine the evolution of intrinsic termination across the bacterial kingdom. A first collaborative study provides background and motivation in the form of a method for identifying Rho-independent terminators using covariance models built from deep alignments of experimentally-verified terminators from \textit{Escherichia coli} and \textit{Bacillus subtilis}. In the course of the development of this method I discovered a novel putative intrinsic terminator in \textit{Mycobacterium tuberculosis}. In the final chapter, I extend this approach to de novo discovery of intrinsic termination motifs across the bacterial phylogeny. I present evidence for lineage-specific variations in canonical Rho-independent terminator composition, as well as discover seven non-canonical putative termination motifs. Using a collection of publicly available RNA-seq datasets, I provide evidence for the function of some of these elements as \textit{bona fide} transcriptional attenuators.

\end{abstracts}
%\end{abstractlongs}


% ----------------------------------------------------------------------


%%% Local Variables: 
%%% mode: latex
%%% TeX-master: "../thesis"
%%% End: 
