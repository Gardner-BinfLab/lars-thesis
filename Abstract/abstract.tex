
% Thesis Abstract -----------------------------------------------------


%\begin{abstractslong}    %uncommenting this line, gives a different abstract heading
\begin{abstracts}        %this creates the heading for the abstract page

Before the advent of the high throughput methods of molecular biology, the world of non-coding RNAs contained mostly RNAs involved in protein synthesis. With the help of the recent advances in the field of genomics numerous types of non-coding RNAs have been identified and their diverse biogenesis and functions are still being revealed. The aim of the following studies was to investigate several non-coding RNAs with methods of computational biology.

	A class of small RNAs called the PIWI-interacting RNAs or piRNAs protect the genome of mouse embryonic germ cells from transposable elements.  However, the function and biogenesis of the adult, pachytene piRNAs is largely unknown. The first aim of my doctorate program was to investigate transcription and regulation of pachytene piRNAs. A set of experiments on testis and liver tissues have been performed in collaboration with the Duncan Odom group (Cancer Research Institute, UK) to explore the transcriptomic and epigenetic characteristics of piRNA genomic loci. It was discovered that the piRNA precursors are mainly transcribed by RNA polymerase II, contain epigenetic hallmarks of protein-coding genes and are regulated by the transcription factor MYBL1. Furthermore, the evolutionary conserved mechanism of piRNA biogenesis was identified that requires transcription of piRNA precursor loci from convergent promotors. 

Unlike the pachytene piRNAs, the embryonic piRNAs have a clear function in silencing of transposable elements. These piRNAs prime the PIWI proteins to cut the transposon transcripts and methylate their genomic loci. The current model of piRNA biogenesis presumes an amplification step that is performed by the coordinated activity of mouse Piwi homologues Mili and Miwi2. The second aim of my doctoral studies was  to validate the enzymatic properties of both of these proteins. To that end, a set of transgenic mice were established with the Donal O'Carroll group (EMBL-Monterotondo) and their phenotype was investigated on the piRNA level. This results pointed to the conclusion that amplification is performed solely by the Mili protein and suggested a necessary modification of the current model of piRNA biogenesis.

Similarly to the embryonic piRNAs, a class of small RNAs termed microRNAs also silence cellular transcripts. These RNAs reduce the expression levels of a multitude of protein-coding transcripts at once. The prediction of miRNA targets has been attempted through methods of computational and molecular biology, without a definite solution. Recently, the Sylamer algorithm was developed that circumvents the miRNA target identification by detection of the miRNA influence on expression profiles of genes. As the last part of my doctorate program the SylArray web server was built based on the Sylamer algorithm. The web server allows researchers from a broad area of expertise to perform fast detection, statistical analysis and visualisation of small RNA signatures in gene expression datasets.

\end{abstracts}
%\end{abstractlongs}


% ----------------------------------------------------------------------


%%% Local Variables: 
%%% mode: latex
%%% TeX-master: "../thesis"
%%% End: 
