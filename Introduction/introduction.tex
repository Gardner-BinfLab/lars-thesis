%%% Thesis Introduction --------------------------------------------------
\chapter{Introduction}
\markboth{Introduction}{Introduction}
\ifpdf
    \graphicspath{{Introduction/IntroductionFigs/PNG/}{Introduction/IntroductionFigs/PDF/}{Introduction/IntroductionFigs/}}
\else
    \graphicspath{{Introduction/IntroductionFigs/EPS/}{Introduction/IntroductionFigs/}}
\fi

Bacteria possess a remarkable ability to adapt. This ability has allowed bacteria colonize almost every environ on Earth, from deep sea hydrothermal vents \parencite{Jorgensen1992} to cryogenic brine lakes \parencite{Murray2012} to animal hosts \parencite{Finlay1997}. Indeed, the ability of bacteria to establish symbiotic relationships with host cells was a critical step in the origin of so-called ``higher" eukaryotic life \parencite{Sagan1967}. While the origins of some adaptations, such as the differing bauplans 

Several factors likely contribute to this ability, including large population sizes, wide-spread homologous recombination, and a capacity for horizontal gene transfer. These factors, particularly homologous recombination, 