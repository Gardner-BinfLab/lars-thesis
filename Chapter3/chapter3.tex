% \pagebreak[4]
% \hspace*{1cm}
% \pagebreak[4]
% \hspace*{1cm}
% \pagebreak[4]
%\usepackage[round,colon,authoryear]{natbib}

\chapter{A pipeline for the analysis of TraDIS experiments, with an application to {\it Salmonella} macrophage invasion}
\label{sec:chapterPingpong}
\ifpdf
    \graphicspath{{Chapter3/Chapter3Figs/EPS/}{Chapter3/Chapter3Figs/}}
\fi

\textit{Section 3.3 describes a collaborative study with Gemma C. Langridge (Pathogen Genomics, Wellcome Trust Sanger Institute). Gemma performed all laboratory experiments described in this chapter.}

\section{Introduction}

In the previous chapter, I described the results of a study predicting and comparing the genes required for robust growth of two \textit{Salmonella} serovars in standard laboratory media. While this revealed interesting aspects of \textit{Salmonella} biology, linking these findings to \textit{Salmonella}'s infective niche in the human host is difficult. However, transposon-insertion sequencing can be used to interrogate infective conditions directly (reviewed previously in section 1.5): by comparing libraries passed through a condition of interest to control libraries, we can determine the genomic regions involved in survival in that condition. In section 3.3, I describe results of such an experiment examining \textit{Salmonella} invasion of human macrophage and establishment of the \textit{Salmonella} containing vacuole (SCV\nomenclature[Z]{SCV}{\textit{Salmonella} containing vacuole}). First, however, I will describe the computational methods used to process this data.

\section{Methods}

Determining conditional gene fitness presents a somewhat different problem to that addressed in the previous chapter, predicting and comparing ``essential'' genes. In predicting gene essentiality, we had a single time point representing the initial growth of the library on rich media, while in identifying conditional gene fitness we are always comparing changes in mutant fitness with respect to fitness in a baseline condition. In some ways, this makes the problem of identifying genes with strong fitness effects easier: as we 

\subsection{Mapping insertion sites}

\subsection{Defining genomic features}

\subsection{Quality control}

\subsection{Inter-library normalization}

\subsection{Identifying fitness effects}

\subsection{Functional analysis of gene sets that affect fitness}

\section{Application: \textit{Salmonella enterica} macrophage invasion}

\subsection{Background: the {\it Salmonella} containing vacuole}

\subsection{Experimental methods}

\textit{All experiments were performed by Gemma Langridge. They are described here briefly for completeness; in-depth descriptions of the experiments are available in \parencite{Langridge2010}}


\subsection{Results}

\subsection{Other applications}