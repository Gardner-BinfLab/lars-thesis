% \pagebreak[4]
% \hspace*{1cm}
% \pagebreak[4]
% \hspace*{1cm}
% \pagebreak[4]
%\usepackage[round,colon,authoryear]{natbib}

\chapter{A comparison of dense transposon insertion libraries in the \textit{Salmonella} serovars Typhi and Typhimurium}
\label{sec:chapterPingpong}
\ifpdf
    \graphicspath{{Chapter2/Chapter2Figs/EPS/}{Chapter2/Chapter2Figs/}}
\fi

\textit{This chapter is a modified version of the previously published article ``A comparison of dense transposon insertion libraries in the \emph{Salmonella} serovars Typhi and Typhimurium'' \parencite{Barquist2013a}. This work is a result of collaboration with Gemma C. Langridge (Pathogen Genomics, Wellcome Trust Sanger Institute), who constructed the \emph{Salmonella} Typhimurium transposon mutant library and contributed to a draft manuscript. In particular, portions of the analyses in sections 2.3.1-3 have their origins in \textcite{Langridge2010}, though have been significantly elaborated on here.}

\section{Introduction}

\textit{Salmonella enterica} subspecies \textit{enterica} serovars Typhi ({\it S.} Typhi\footnote{Note that the complicated {\it Salmonella} taxonomy and nomenclature make abbreviation difficult (and at times contentious). Here I have adopted the practice of referring to individual serovars as {\it S.} Serovar once they have been introduced, following the advice of \textcite{Brenner2000}.}) and Typhimurium ({\it S.} Typhimurium) are important, closely related, human pathogens with very different lifestyles. In this chapter, I describe a study comparing dense transposon insertion libraries created in these two serovars. The results of this study demonstrate that orthologous genes can have dramatically different effects on the fitness of recently diverged organisms in rich media. These differences in fitness effects are indicative of changes in the network architecture of the cell which may partially underlie the dramatically different diseases caused by each organism and their different host ranges. Additionally, {\it S.} Typhimurium has served as a model organism for the discovery and functional characterization of ncRNAs. Comparing ncRNA requirements between it and a closely related serovar provides a glimpse of the functional evolution of non-coding regulatory networks.

\subsection{The genus {\it Salmonella}}

\textit{Salmonella} is a Gram-negative, $\gamma$-proteobacterial genus within the order Enterobacteriales, consisting of two species: \textit{Salmonella enterica} and \textit{Salmonella bongori}, though a third species, \textit{Salmonella subterranea}, has recently been proposed \parencite{Shelobolina2004}. Based on phylogenetic analyses of 16S and conserved amino acid sequences, {\it Salmonella} is most closely related to the genuses \textit{Escherichia}, \textit{Shigella}, and \textit{Citrobacter} \parencite{Paradis2005, Pham2007, Wu2009}. Molecular clock analyses suggest that \textit{Salmonella} and \textit{Escherichia} shared a common ancestor between 100 and 160 million years ago \parencite{Ochman1987, Doolittle1996}. During the time since their divergence \textit{Escherichia} has become established as a mammalian gut commensal, though multiple independent origins of the \textit{Shigella} and other pathogenic phenotypes within the genus show that a disease phenotype can be developed fairly easily through the horizontal acquisition of virulence determinants and the silencing of anti-virulence loci \parencite{Kaper2004,Prosseda2012}. Despite sharing the majority of their genomes with \textit{Escherichia} and having broadly similar metabolic capabilities \parencite{AbuOun2009}, the salmonellae exist primarily as pathogens, though are possibly commensal in some reptiles \parencite{Mermin2004, Bauwens2006}. 

\begin{figure}[htp]
\begin{center}
\includegraphics[width=14cm]{bongori.png}
\caption[Genomic acquisitions in the evolution of the salmonellae]{\textbf{Genomic acquisitions in the evolution of the salmonellae.} Traits shared by the common ancestor are depicted in blue; those unique to {\it S. bongori} are shown in red and those unique to {\it S. enterica} subspecies {\it enterica} serovar Typhi in green. Arrows, {\it Salmonella} Pathogenicity Islands (SPIs); extended ovals, fimbriae; circles, effectors; small ovals and needle complexes, secretion systems. Metabolic pathways: lines, enzymatic reactions; open squares, carbohydrates; ovals, pyrimidines; open circles, other substrates; filled shapes, phosphorylated. Novel effectors acquired by {\it S. bongori} are secreted by the type III secretion system encoded on SPI-1. SPI-3a and 3b carry the same genes in both organisms but are fused into one island in {\it S.} Typhi. SPI-5a also carries the same genes in both organisms, but a further 3 kb (termed SPI-5b) has fused to SPI-5a in {\it S.} Typhi. *indicates a pseudogene. Reproduced from \textcite{Fookes2011} under a Creative Commons Attribution License (CCAL). 
} 
\label{fig:bongori}
\end{center}
\end{figure}

The difference in dominant phenotype between {\it Escherichia} and {\it Salmonella} appears to be largely due to the acquisition of virulence determinants which opened new niches to ancestral salmonellae (see figure \ref{fig:bongori}). Interestingly, \textit{S. subterranea} has been described as a soil isolate capable of uranium reduction \parencite{Shelobolina2004} and may represent a pre-pathogenic branching of the lineage, though the species has not been thoroughly characterized and it is not clear if this strain is truly a member of the genus. Many of the virulence determinants characteristic of the salmonellae are encoded on large genomic islands with sizes between $\sim$6 and 140 kilobases, termed \textit{Salmonella} Pathogenicity Islands (SPIs) \parencite{Hensel2004}. These islands encode a diverse array of pathogenicity-related functions including secretion systems, toxins, antibiotic resistances, and lipopolysaccharide (LPS)\nomenclature[Z]{LPS}{Lipopolysaccharide} and capsular modifications. In particular, the acquisition of SPI-1, encoding a type 3 secretion system (T3SS), and various fimbriae by the ancestral \textit{Salmonella} likely enabled invasion of cells in the intestinal epithelium and escape from competition with other members of the gut microbiota \parencite{Baumler1997}. \textit{S. bongori} appears to have only acquired a single additional SPI since its divergence from \textit{S. enterica} and likely retains a lifestyle more similar to the ancestral \textit{Salmonella}, though there is evidence for additional adaptation to its niche in the reptilian gut \parencite{Fookes2011}. 

\textit{S. enterica} meanwhile has diversified into 6 distinct subspecies: {\it enterica}, {\it salamae}, {\it arizonae}, {\it diarizonae}, {\it houtenae}, and {\it indica}. These subspecies are further divided into over 2000 serovars based on the cell-surface O, flagellar H, and capsular Vi antigens \parencite{Grimont2007}. The acquisition of SPI-2, involved in survival inside macrophages and an enabling factor for systemic infection \parencite{Kuhle2004, Abrahams2006}, by the ancestral {\it S. enterica} is thought to have been a driving force in this diversification \parencite{Baumler1997}.  Subspecies besides {\it enterica} are thought to be primarily restricted to cold-blooded animals \parencite{Baumler1997}, though sporadic reports of zoonotic disease show these subspecies are capable of transiently infecting the mammalian gut under certain conditions \parencite{Mermin2004,Hilbert2012}. However, here I will be primarily concerned with the subspecies {\it enterica} and its adaptation to the mammalian, and more specifically human, host.

\subsection{Host adaptation and restriction}

Bacterial adaptation to a pathogenic lifestyle is a complex process involving both the acquisition of virulence factors and gene loss through both passive decay and positive selection \parencite{Pallen2007, Prosseda2012}. In the previous section I discussed how the acquisition of SPI-1 and -2, among other factors, have enabled \textit{S. enterica} subspecies \textit{enterica} to establish a niche in the mammalian gut. Access to this new niche has enabled serovars of subspecies {\it enterica} to explore a range of pathogenic modalities. The most common form of disease caused by {\it enterica} serovars is a self-limiting gastroenteritis, exemplified by the serovars Typhimurium and Enteriditis \parencite{Santos2009}. These serovars can infect a wide range of mammals and birds, but are only capable of causing serious disease in the very young \parencite{Baumler1998}, and are generally thought to exhibit a phenotype similar to the ancestral {\it enterica}.

A number of subspecies {\it enterica} serovars have adapted to causing invasive disease in specific organisms. These include Typhi and Paratyphi in humans, Dublin in cattle, Gallinarum in chickens, Abortusovis in sheep, Choleraesuis in pigs, and Abortusequi in horses. These adaptations appear to be the result of the acquisition of host-specific virulence factors \parencite{Baumler1998}. Interestingly, those serovars associated with the most severe forms of disease appear to be most highly restricted in terms of host range. This appears to be the result of three processes: positive selection against anti-virulence loci \parencite{Pallen2007, Prosseda2012}, and two more passive processes termed ``use it or lose it'' and ``use it, but lose it anyway'' by \textcite{Moran2002}. 

Selection against anti-virulence loci presumably occurs during host-adaptation, and generally involved the loss of loci that provoke an antigenic response or interfere with the infective process. Once a bacterium has escaped competition in the gut microbiota and gained access to a rich intracellular niche through horizontal acquisitions, the ``use it or lose it'' principle leads to the loss of metabolic pathways no longer required in this environment presumably due to the lifting of selective pressure for their maintenance. The ``use it, but lose it anyway'' principle is a consequence of the severe bottleneck imposed by adaptation to a particular host, which will often drastically reduce the effective population size of the bacterium. This can cause fixation of inactivating mutations in potentially beneficial genes simply as an accident of the adaptive process. Together these processes may eventually prevent the bacterium from living independently of its host; particularly extreme examples are \textit{Mycobacterium leprae} with its thousands of inactivated pseudogenes \parencite{Cole2001}, \textit{Mycoplasma} species with their highly reduced genomes \parencite{Fraser1995}, and most strikingly the endosymbiont-derived mitochondria and plastid organelles \parencite{Sagan1967, Andersson1998}. While no \textit{Salmonella} serovars appear to have been subject to this degree of genome degradation, it is not unusual for as much as 7\% of the protein-coding genes of host-restricted serovars to be inactivated \parencite{Parkhill2001,Thomson2008, Holt2009, McClelland2004}.

The serovars of \textit{S. enterica} subspecies \textit{enterica} exhibit a spectrum of pathogenic lifestyles, from low-pathogenicity and wide host range to high-pathogenicity and narrow host range. Recent studies examining host adaptation of Typhimurium strains to immunocompromised populations \parencite{Feasey2012, Okoro2012} demonstrate that the process of host-adaptation is both on-going and highly relevant to human health. In this study, we have used transposon-insertion sequencing to examine two recently diverged (circa 50,000 years ago \parencite{Kidgell2002}) serovars at extreme ends of this pathogenicity spectrum: Typhi and Typhimurium. 

\subsection{Serovars Typhi and Typhimurium}

\textit{Salmonella enterica} subspecies \textit{enterica} serovars Typhi ({\it S.} Typhi) and Typhimurium ({\it S.} Typhimurium ) are important human pathogens with distinctly different lifestyles. {\it S.} Typhi is host-restricted to humans and causes typhoid fever. This potentially fatal systemic illness affects at least 21 million people annually, primarily in developing countries \parencite{Crump2004, Bhutta2009, Kothari2008}, and is capable of colonizing the gall bladder creating asymptomatic carriers; such individuals are the primary source of this human restricted infection, exemplified by the case of ``Typhoid Mary'' \parencite{Soper1939}. Mary Mallon was an Irish-American cook in New York City at the turn of the twentieth century, and an (at least initially) unwitting carrier of Typhi. A series of typhoid outbreaks were traced to her by city public health authorities. She was offered removal of her gall bladder, which she refused, and was ordered to refrain from working as a cook following release from three years of quarantine. After a number of additional outbreaks -- including several deaths -- were traced to Mary, who had continued working as a cook under a pseudonym, she was involuntarily quarantined on North Brother Island in the East River for 23 years until her death.

{\it S.} Typhimurium, conversely, is a generalist, causing relatively mild disease in a wide range of mammals and birds in addition to being a leading cause of foodborne gastroenteritis in human populations. Control of {\it S.} Typhimurium infection in livestock destined for the human food chain is of great economic importance, particularly in swine and cattle \parencite{CDC2009, Majowicz2010}. Additionally, {\it S.} Typhimurium causes an invasive disease in mice, which has been used extensively as a model for pathogenicity in general and human typhoid fever specifically \parencite{Santos2001}.

Despite this long history of investigation, the genomic factors that contribute to these differences in lifestyle remain unclear. Over 85\% of predicted coding sequences are conserved between the two serovars in sequenced genomes of multiple strains \parencite{McClelland2001, Parkhill2001, Holt2008, Deng2003}. The horizontal acquisition of both plasmids and pathogenicity islands during the evolution of the salmonellae is believed to have impacted upon their disease potential. A 100kb plasmid, encoding the {\it Salmonella} plasmid virulence (SPV)\nomenclature[Z]{SPV}{{\it Salmonella} plasmid virulence (genes)} genes, is found in some {\it S.} Typhimurium strains and contributes significantly towards systemic infection in animal models \parencite{Gulig1987, Gulig1993}. {\it S.} Typhi is known to have harbored IncHI1 plasmids conferring antibiotic resistance since the 1970�s \parencite{Phan2009}, and there is evidence that these strains present a higher bacterial load in the blood during human infection \parencite{Wain1998}. Similar plasmids have been isolated from {\it S.} Typhimurium \parencite{Datta1962,Holt2007,Cain2012}. {\it Salmonella} pathogenicity islands 1 and 2 are common to all {\it Salmonella enterica} subspecies, and are required for invasion of epithelial cells (reviewed in \textcite{Darwin1999}) and survival inside macrophages respectively \parencite{Ochman1996,Shea1996,Kuhle2004, Abrahams2006}. {\it S.} Typhi additionally incorporates SPI-7 and SPI-10, which contain the Vi surface antigen and a number of other putative virulence factors \parencite{Pickard2003,Seth-Smith2008,Townsend2001}. 

Acquisition of virulence determinants is not the sole explanation for the differing disease phenotypes displayed in humans by {\it S.} Typhimurium and {\it S.} Typhi; genome degradation is an important feature of the {\it S.} Typhi genome, in common with other host-restricted serovars such as {\it S.} Paratyphi A (humans) and {\it S.} Gallinarum (chickens). In each of these serovars, pseudogenes account for 4-7\% of the genome \parencite{Parkhill2001,Thomson2008, Holt2009, McClelland2004}. Loss of function has occurred in a number of {\it S.} Typhi genes that have been shown to encode intestinal colonisation and persistence determinants in {\it S.} Typhimurium \parencite{Kingsley2003}. Numerous sugar transport and degradation pathways have also been interrupted \parencite{Parkhill2001}, but remain intact in {\it S.} Typhimurium, potentially underlying the restricted host niche occupied by {\it S.} Typhi.

In addition to its history as a model organisms for pathogenicity, {\it S.} Typhimurium has recently served as a model organism for the elucidation of non-coding RNA (ncRNA) function \parencite{Vogel2009a}. These include cis-acting switches, such as RNA-based temperature and magnesium ion sensors \parencite{Waldminghaus2007, Cromie2006}, together with a host of predicted metabolite-sensing riboswitches. Additionally, a large number of trans-acting small RNAs (sRNAs) have been identified within the {\it S.} Typhimurium genome \parencite{Kroger2012}, some with known roles in virulence \parencite{Hebrard2012}. These sRNAs generally control a regulon of mRNA transcripts through an antisense binding mechanism mediated by the protein Hfq in response to stress. The functions of these molecules have generally been explored in either {\it S.} Typhimurium or {\it E. coli}, and it is unknown how stable these functions and regulons are over evolutionary time \parencite{Richter2012}.

Transposon mutagenesis has previously been used to assess the requirement of particular genes for cellular viability. The advent of next-generation sequencing has allowed simultaneous identification of all transposon insertion sites within libraries of up to 1 million independent mutants (reviewed in \textcite{Barquist2013}; see also the previous chapter), enabling us to answer the basic question of which genes are required for {\it in vitro} growth with extremely fine resolution. By using transposon mutant libraries of this density, which in {\it S.} Typhi represents on average $>$ 80 unique insertions per gene \parencite{Langridge2009a}, shorter regions of the genome can be interrogated, including ncRNAs \parencite{Christen2011}. In addition, once these libraries exist, they can be screened through various selective conditions to further reveal which functions are required for growth/survival.

Using Illumina-based transposon directed insertion-site sequencing (TraDIS \parencite{Langridge2009a}) with large mutant libraries of both {\it S.} Typhimurium and {\it S.} Typhi, we investigated whether these salmonellae require the same protein-coding and non-coding RNA (ncRNA) gene sets for competitive growth under laboratory conditions, and whether there are differences which reflect intrinsic differences in the pathogenic niches these bacteria inhabit.

\section{Materials and Methods}

\textit{Gemma Langridge created the \emph{S.} Typhimurium library described here, and performed all the laboratory experiments described here. Duy Phan and Keith Turner created the \emph{S.} Typhi library. Duy Phan and Gemma Langridge performed the read mapping.}

\subsection{Strains}

{\it S.} Typhimurium strain SL3261 contains a deletion relative to the parent strain, SL1344, was used to generate the large transposon mutant library. The 2166bp deletion ranges from 153bp within {\it aroA} (normally 1284bp) to the last 42bp of {\it cmk}, forming two pseudogenes and deleting the intervening gene SL0916 completely. For comparison, we utilized our previously generated {\it S.} Typhi Ty2 transposon library \parencite{Langridge2009a}.

\subsection{Annotation}

For {\it S.} Typhimurium strain SL3261, I used feature annotations drawn from the SL1344 genome (EMBL-Bank accession FQ312003.1), ignoring the deleted {\it aroA}, {\it ycaL}, and {\it cmk} genes. I re-analyzed the {\it S.} Typhi Ty2 transposon library with features drawn from an updated genome annotation (EMBL-Bank accession AE014613.1.) I supplemented the EMBL-Bank annotations with non-coding RNA annotations drawn from Rfam 10.1 \parencite{Burge2013}, \textcite{Sittka2008}, \textcite{Chinni2010}, \textcite{Raghavan2011}, and \textcite{Kroger2012}. Selected protein-coding gene annotations were supplemented using the HMMER webserver \parencite{Finn2011} and Pfam \parencite{Punta2012}.

\subsection{Creation of {\it S.} Typhimurium transposon mutant library}

{\it S.} Typhimurium was mutagenized using a Tn{\it 5}-derived transposon as described previously (\cite{Langridge2009a}; a detailed protocol is available in \cite{Langridge2010}). Briefly, the transposon was combined with the EZ-Tn{\it 5} transposase (Epicenter, Madison, USA) and electroporated into {\it S.} Typhimurium. Transformants were selected by plating on LB agar containing 15 $\mu$g/mL kanamycin and harvested directly from the plates following overnight incubation. A typical electroporation experiment generated a batch of between 50,000 and 150,000 individual mutants. 10 batches were pooled together to create a mutant library comprising approximately 930,000 transposon mutants.

\subsection{DNA manipulations and sequencing}
Genomic DNA was extracted from the library pool samples using tip-100g columns and the genomic DNA buffer set from Qiagen (Crawley, UK). DNA was prepared for nucleotide sequencing as described previously \parencite{Langridge2009a}. Prior to sequencing, a 22 cycle PCR was performed as previously described \parencite{Langridge2009a}. Sequencing took place on a single end Illumina flowcell using an Illumina GAII sequencer, for 36 cycles of sequencing, using a custom sequencing primer and 2x Hybridization Buffer \parencite{Langridge2009a}. The custom primer was designed such that the first 10 bp of each read was transposon sequence.  

\subsection{Sequence analysis}
The Illumina FASTQ sequence files were parsed for 100\% identity to the $5'$ 10bp of the transposon (TAAGAGACAG). Sequence reads which matched were stripped of the transposon tag and subsequently mapped to the {\it S.} Typhimurium SL1344 or {\it S.} Typhi Ty2 chromosomes using Maq version maq-0.6.8 \parencite{Li2008}. Approximately 12 million sequence reads were generated from the sequencing run which used two lanes on the Illumina flowcell. Precise insertion sites were determined using the output from the Maq mapview command, which gives the first nucleotide position to which each read mapped. The number and frequency of insertions mapping to each nucleotide in the appropriate genome was then determined. 

\subsection{Statistical analysis of required genes}

The number of insertion sites for any gene is dependent upon its length, so the values were made comparable by dividing the number of insertion sites by the gene length, giving an ``insertion index'' for each gene. As before \parencite{Langridge2009a} the distribution of insertion indices was bimodal, corresponding to the required (mode at 0) and non-required distributions (See Figure \ref{fig:gamma}). I fitted gamma distributions for the two modes using the R MASS library (http://www.r-project.org). $Log_2$-likelihood ratios (LLR)\nomenclature[Z]{LLR}{$Log_2$-likelihood ratios} were calculated between the required and non-required distributions and I called a gene required if it had an LLR of less than -2, indicating it was at least 4 times more likely according to the required model than the non-required model. ``Non-required'' genes were assigned for an LLR of greater than 2. Genes falling between the two thresholds were considered ``ambiguous'' for the purpose of this analysis. This procedure lead to genes being called as required in {\it S.} Typhimurium when their insertion index was less than 0.020, or 1 insertion in every 50 bases, and ambiguous between 0.020 and 0.027. The equivalent cut-offs for the {\it S.} Typhi library are 0.0147 and 0.0186, respectively.

\begin{figure}[htp]
\begin{center}
\includegraphics[width=14cm]{typhi_gamma}
\caption[The distribution of gene-wise insertion indexes in {\it S.} Typhi]{\textbf{The distribution of gene-wise insertion indexes in {\it S.} Typhi.} Bars report the density of genes with insertion indexes within each range, black lines show gamma distributions fitted to the required (left, mode at 0) and non-required (right) peaks, and red lines report associated LLR-based cut-offs for calling gene ambiguity (left) and requirement (right). The distribution of insertion indexes in {\it S.} Typhimurium is similar, though with a wider separation between the required and non-required peaks due to the higher insertion density attained. 
} 
\label{fig:gamma}
\end{center}
\end{figure}

I calculated a p-value for the observed number of insertion sites per gene using a Poisson approximation with rate $R = \frac{N}{G}$ where $N$ is the number of unique insert sites (549,086) and $G$ is the number of bases in the genome (4,878,012). The p-value for at least $X$ consecutive bases without an insert site is $e^{(-RX)}$, giving a 5\% cut-off at 27 bp and a 1\% cut-off at 41 bp.

For every gene $g$ with $n_{g,A}$ reads observed in {\it S.} Typhi and $n_{g,B}$ reads observed in {\it S.} Typhimurium, I calculated the log2 fold change ratio $S_{g,A,B} = log2 (\frac{n_{g,A}+100}{n_{g,B}+100})$. The correction of 100 reads smoothes out the high scores for genes with very low numbers of observed reads. I fitted a normal distribution to the mode +/- 2 sample standard deviations of the distribution of $S_{A,B}$, and calculated p-values for each gene according to the fit. I considered genes with a p-value of 0.05 or less under the fitted normal distribution to be uniquely required by one serovar.

\section{Results and Discussion}

\subsection{TraDIS assay of every {\it Salmonella} Typhimurium protein-coding gene}
Approximately 930,000 mutants of {\it S.} Typhimurium were generated using a Tn{\it 5}-derived transposon. 549,086 unique insertion sites were recovered from the mutant library using short-read sequencing with transposon-specific primers. This is a substantially higher density than the 371,775 insertions recovered from {\it S.} Typhi previously \parencite{Langridge2009a}. The {\it S.} Typhimurium library contains an average of one insertion every 9bp, or over 100 unique inserts per gene (figure  \ref{fig:genome}). The large number of unique insertion sites allowed every gene to be assayed; assuming random insertion across the genome, a region of 41bp without an insertion was statistically significant (P $<$ 0.01).  As previously noted in {\it S.} Typhi, the distribution of length-normalized insertions per gene is bimodal (see figure \ref{fig:gamma}), with one mode at 0. We interpret genes falling in to the distribution around this mode as being required for competitive growth within a mixed population under laboratory conditions (hereafter ``required''). Of these, 57 contained no insertions whatsoever and were mostly involved in core cellular processes (see table \ref{tab:core}).

\begin{figure}[htp]
\begin{center}
\includegraphics[width=14cm]{genome}
\caption[Genome-wide transposon mutagenesis of {\it S.} Typhimurium]{\textbf{Genome-wide transposon mutagenesis of {\it S.} Typhimurium.} A) Circular plot showing gene content, distribution of required genes, and insertion density along the {\it S.} Typhimurium chromosome. The outer scale is marked in megabases. Circular tracks range from 1 (outer track) to 6 (inner track). Track 1, all forward-strand genes (color-coded according to function: dark blue, pathogenicity/adaptation; black, energy metabolism; red, information transfer; dark green, membranes/surface structures; cyan, degradation of macromolecules; purple, degradation of small molecules; yellow, central/intermediary metabolism; light blue, regulators; pink, phage/IS elements; orange, conserved hypothetical; pale green, unknown function; brown, pseudogenes.); track 2, all reverse-strand genes (color-coded as for forward-strand genes); track 3, {\it S.} Typhimurium required genes (red); track 4, 56 genes required by {\it S.} Typhimurium but not by {\it S.} Typhi (dark blue, see also table 1); track 5, transposon insertion density; track 6, GC bias ($\frac{G-C}{G+C}$), khaki indicates values $>$1; purple $<$1.
} 
\label{fig:genome}
\end{center}
\end{figure}

There was a bias in the frequency of transposon insertion towards the origin of replication. This likely occurred as the bacteria were in exponential growth phase immediately prior to transformation with the transposon. In this phase of growth, multiple replication forks would have been initiated, meaning genes closer to the origin were in greater copy number and hence more likely to be a target for insertion. We also observed a bias for transposon insertions in A+T rich regions, as was previously observed in the construction of an {\it S.} Typhi mutant library \parencite{Langridge2009a}. However, the insertion density achieved is sufficient to discriminate between required and non-required genes easily. As was first seen in {\it S.} Typhi \parencite{Langridge2009a}, we observed transposon insertions into genes upstream of required genes in the same operon, suggesting that most insertions do not have polar effects leading to the inactivation of downstream genes.

Analysis of the {\it S.} Typhimurium mutant library allowed us to identify 353 coding sequences required for growth under laboratory conditions, and 4,112 non-required coding sequences (see {\color{red} need to put supp data in appendix}). We were unable to assign 65 genes to either the required or non-required category. 60 of these genes, which we will refer to as ``ambiguous'', had log-likelihood ratios (LLRs) between -2 and 2. The final 5 unassigned genes had lengths less than 60 bases, and they were removed from the analysis. All other genes contained enough insertions or were of sufficient length to generate credible LLR scores. Thus, every gene was assayed and we were able to draw conclusions for 98.7\% of the coding genome in a single sequencing run (figure \ref{fig:genome}).

\subsection{Cross-species comparison of genes required for growth}

\begin{figure}[htp]
\begin{center}
\includegraphics[width=14cm]{venn}
\caption[Comparison of required genes]{\textbf{Comparison of required genes.} Venn diagrams showing (A) the overlap of all genes (outer circles, light colors) and required genes (inner circles, dark colors) between {\it S.} Typhimurium and {\it S.} Typhi (excluding genes required in one serovar only which do not have significantly different read-counts). Black numbers refer to all genes, white numbers to required genes. (B) the overlap of all required genes between {\it S.} Typhimurium (blue), {\it S.} Typhi (green) and {\it E. coli} (purple). White numbers refer to genes with Keio essentiality scores $>=$ 0.5.
} 
\label{fig:venn}
\end{center}
\end{figure}

Gene essentiality has previously been assayed in {\it S.} Typhimurium using insertion-duplication mutagenesis. \textcite{Knuth2004} estimated 490 genes are essential to growth in clonal populations, though 36 of these have subsequently been successfully deleted \parencite{Santiviago2009}. While TraDIS assays gene requirements after a brief period of competitive growth on rich media, we identify a smaller required set than \textcite{Knuth2004} of approximately 350 genes in each serovar, closer to current estimates of approximately 300 essential genes in {\it E. coli} \parencite{Baba2006}. 

To demonstrate that TraDIS does identify genes known to have strong effects on growth, as well as to test our predictive power for determining gene essentiality, we compared our required gene sets in {\it S.} Typhimurium and {\it S.} Typhi to essential genes determined by systematic single-gene knockouts in the {\it Escherichia coli} K-12 Keio collection \parencite{Baba2006}. We identified orthologous genes in the three data sets by best reciprocal FASTA\nomenclature[Z]{FASTA}{Fast alignment} hits exhibiting over 30\% sequence identity over at least 80\% of the amino acid sequence. Required orthologous genes identified in this manner share a significantly higher average percent sequence identity with their {\it E. coli} counterparts than expected for a random set of orthologs, at $\sim$94\% identity as compared to $\sim$87\% for all orthologous genes. In 100,000 randomly chosen gene sets of the same size as our required set we did not find a single set where the average shared identity exceeded 90\%, indicating that required genes identified by TraDIS are more highly conserved at the amino acid level than other orthologous protein coding sequences.

\textcite{Baba2006} have defined an essentiality score for each gene in {\it E. coli} based on evidence from four experimental techniques for determining gene essentiality: targeted knock-outs using $\lambda$-red\nomenclature[G]{$\lambda$}{Phage lambda} mediated homologous recombination, genetic footprinting \parencite{Gerdes2003,Tong2004}, large-scale chromosomal deletions \parencite{Hashimoto2005}, and transposon mutagenesis \parencite{Kang2004}.  Scores range from -4 to 3, with negative scores indicating evidence for non-essentiality and positive scores indicating evidence for essentiality. Comparing the overlap between essential gene sets in {\it E. coli}, {\it S.} Typhi, and {\it S.} Typhimurium, we find a set of 228 {\it E. coli} genes which have a Keio essentiality score of at least 0.5 (i.e. there is evidence for gene essentiality; See Figure \ref{fig:venn}.) that have TraDIS-predicted required orthologs in both {\it S.} Typhi and {\it S.} Typhimurium, constituting $\sim$85\% of {\it E. coli} genes with evidence for essentiality indicating that gene requirements are largely conserved between these genera. Including orthologous genes that are only predicted to be essential by TraDIS in {\it S.} Typhi or {\it S.} Typhimurium raises this figure to nearly 93\%. The majority of shared required genes between all three bacteria are responsible for fundamental cell processes, including cell division, transcription and translation. A number of key metabolic pathways are also represented, such as fatty acid and peptidoglycan biosynthesis (Table \ref{tab:core}). A recent study in the $\alpha$-proteobacteria {\it Caulobacter crescentus} reported 210 shared essential genes with {\it E. coli}, despite {\it C. crescentus} sharing less than a third as many orthologous genes with {\it E. coli} as {\it Salmonella} serovars \parencite{Christen2011}. This suggests the existence of a shared core of approximately 200 essential proteobacterial genes, with the comparatively rapid turnover of 150 to 250 �non-core� lineage-specific essential genes.  

% Table generated by Excel2LaTeX from sheet 'Sheet1'
%
\begin{landscape}
\begin{table}
   \small
   \centering
   \noindent
    \caption[Core genome functions in \emph{S.} Typhimurium]{\textbf{Core genome functions in \emph{S.} Typhimurium.} Protein-coding genes providing fundamental biological functions in \emph{S.} Typhimurium. Genes in bold are required in\emph{S.} Typhi (log-likelihood ratio (LLR) between required and non-required models $<$ -2; see Methods.) * indicates genes ambiguous in \emph{S.} Typhimurium, having a LLR between -2 and 2. }
    \begin{tabular}{     l
    				p{1.5in}
				p{2.9in}
				p{2in}
				}
   
    \\
    \toprule
    \textbf{Biological Process} & \textbf{Sub-process} & \textbf{Required genes} & \textbf{Non-required genes}\\
    \midrule
    Cell division & & \emph{\textbf{ftsALKQWYZ}, minE, mukB, SL2391} & \emph{\textbf{ftsHJNX}*, minCD, sdiA, cedA, sulA}\\
    DNA replications & Polymerases I, II, and III & \emph{\textbf{dnaENQX, holAB}} & \emph{\textbf{polA}B, holC\textbf{D}E}\\
    & Supercoiling & \emph{\textbf{gyrAB, parCE}} & \\
    & Primosome-associated & \emph{\textbf{dnaBCGT, priA, ssb}} & \emph{\textbf{priB}*C, \textbf{rep}}\\
    Transcription & RNA polymerase & \emph{\textbf{rpoABC}} & \\
    & Sigma, elongation, anti- and termination factors & \emph{\textbf{nusBG, rpoDH, rho}} & \emph{\textbf{nusA, rpoE}NS}\\
    Translation & tRNA-synthetases & \emph{\textbf{alaS, argS, asnS, aspS, cysS, glnS, gltX, glyQS, hisS, ileS, leuS, lysS, metG, pheST, proS, serS, thrS, tyrS, valS}} & \emph{trpS, trpS2}\\
    & Ribosome components & \emph{\textbf{rplBCDEFJKLMNOPQRSTUVWXY, rpmABCDHI, rpsABCDEFGHIJKLMNPQST}} & \emph{rplAI, rpmEE2, rpmFH\textbf{J}J2, rps\textbf{OR}*\textbf{U}*V} \\
    & Initiation, elongation, and peptide chain release factors & \emph{\textbf{fusA, infABC, prfAB, tsf, yrdC}} & \emph{efp, prfCH, selB, tuf}\\
    \midrule
    \multicolumn{4}{l}{\textbf{Biosynthetic pathways}}\\
    \midrule
    Peptidoglycan & & \emph{\textbf{murABCDEFGI}} & \emph{ddl, dllA}\\
    Fatty acids & & \emph{\textbf{accABCD, fabABDGHIZ}}&\\
    \bottomrule
    
    \end{tabular}%
    \label{tab:core}%
\end{table}
\end{landscape}



If we make the simplistic assumption that gene essentiality should be conserved between {\it E. coli} and {\it Salmonella}, we can use the overlap of our predictions with the Keio essential genes to provide an estimate of our TraDIS libraries� accuracy for predicting that a gene will be required in a clonal population. Of the 2632 orthologous {\it E. coli} genes which have a Keio essentiality score of less than -0.5 (i.e. there is evidence for gene non-essentiality), only 33 are predicted to be required by TraDIS in both {\it Salmonella} serovars. {\it S.} Typhi contains the largest number of genes predicted by TraDIS to be required with {\it E. coli} orthologs with negative Keio essentiality scores. However, even if we assume these are all incorrect predictions of gene essentiality, this still gives a gene-wise false positive rate (FPR)\nomenclature[Z]{FPR}{False positive rate} of $\sim$2.7\% (81 out of 2981 orthologs) and a positive predictive value (PPV)\nomenclature[Z]{PPV}{Positive predictive value} of $\sim$75\% (247 with essentiality scores greater than or equal to 0.5 out of 328 predictions with some Keio essentiality score.)  Under these same criteria the {\it S.} Typhimurium data set has a lower gene-wise FPR of $\sim$1.6\% (51 out of 3122 orthologs) and a higher PPV of $\sim$82\% (234 out of  285 predictions as before), as we would expect given the library's higher insertion density. In reality these FPRs and PPVs are only estimates; genes which are not essential in {\it E. coli} may become essential in the different genomic context of {\it Salmonella} serovars and vice versa, particularly in the case of {\it S.} Typhi where wide-spread pseudogene formation has eliminated potentially redundant pathways \parencite{Holt2009,McClelland2004}. Additionally, TraDIS will naturally over-predict essentiality in comparison to targeted knockouts, as our library creation protocol necessarily contains a short period of competitive growth between mutants during the recovery from electro-transformation and selection. As a consequence, genes which cause major growth defects, but not necessarily a complete lack of viability in clonal populations, may be reported as `required.'

\subsection{Serovar-specific genes required for growth}
Many of the required genes present in only one serovar encoded phage repressors, for instance the cI\nomenclature[Z]{cI}{Clear 1 ($\lambda$ repressor protein)} proteins of Fels-2/SopE and ST35 (see tables \ref{tab:stm_phage} and \ref{tab:ty_phage}). Repressors maintain the lysogenic state of prophage, preventing transcription of early lytic genes \parencite{Echols1971}. Transposon insertions into these genes will relieve this repression and trigger the lytic cycle, resulting in cell death, and consequently mutants are not represented in the sequenced library. This again broadens the definition of `required' genes; such repressors may not be required for cellular viability in the traditional sense, but once present in these particular genomes, their maintenance is required for continued viability, as long as the rest of the phage remains intact.

% Table generated by Excel2LaTeX from sheet 'Sheet1'
%
\begin{table}
   \tiny
   \centering
   \noindent
    \caption[Phage elements in \textit{S.} Typhimurium]{\textbf{Phage elements in \emph{S.} Typhimurium.} Genomic coordinates determined from annotations in the EMBL annotation for FQ312003 and manual inspection. Repressor domains and architecture were determined using the HMMER webserver \parencite{Finn2011} and Pfam \parencite{Punta2012}. Phage types were determined using repressor sequence similarity searches and information from \textcite{Thomson2004} and \textcite{Kropinski2007}. }
    \begin{tabular}{     m{0.5in}
    				m{0.4in}
				m{0.4in}
				m{0.6in}
				m{1.6in}
				m{0.4in}
				m{0.5in}
				m{0.5in}
				}
   
    \\
     \toprule
    \textbf{Element name} & \textbf{Genomic coordinates} & \textbf{Repressor} & \textbf{Repressor domain(s)} & \textbf{Repressor domain architecture} & \textbf{Predicted active?} & \textbf{Phage type} & \textbf{Required cargo} \\
    \midrule
    Gifsy-2 SLP105 & 1054795 - 1100036 & SL0950 & HTH\_3 (PF01381) &\includegraphics[height=6mm]{rep1}& Yes   & lambdoid & N/A \\
    N/A   & 1913364 - 1925490 & N/A   & N/A   & N/A   & No    & remnant & SL1799 \\
    SLP203 & 2039803 - 2079890 & SL1967 & HTH\_19 (PF12844) and Peptidase\_S24 (PF00717) &\includegraphics[height=6mm]{rep2}& Yes   & P22-like & N/A \\
    Gifsy-1 SLP272 & 2726717 - 2777229 & SL2593 & HTH\_3 (PF01381) &    \includegraphics[height=6mm]{rep3}   & Yes   & lambdoid & SL2549 \\
    SLP281 & 2815382 - 2825915 & SL2633 & 2 X Phage\_CI\_repr (PF07022) &   \includegraphics[height=6mm]{rep4}    & Yes   & degenerate P2-like & N/A \\
    Fels-2 SLP285 & 2855616 - 2888522 & SL2708 & Phage\_CI\_repr (PF07022) &   \includegraphics[height=6mm]{rep5}    & Yes   & P2-like & SL2695 \\
    SLP289 & 2890073 - 2900377 & IsrK RNA (RF01394) & N/A   & N/A   & No    & P4-like & N/A \\
    SLP443 & 4437731 - 4459844 & N/A   & N/A   & N/A   & No    & remnant & SL4132 \\
    \bottomrule    
    \end{tabular}%
    \label{tab:stm_phage}%
\end{table}



{\it S.} Typhimurium and {\it S.} Typhi both contain 8 apparent large phage-derived genomic regions \parencite{Thomson2004, Kropinski2007}.  We were able to identify required repressors in all the intact lambdoid, P2-like, and P22-like prophage in both genomes, including Gifsy-1, Gifsy-2, and Fels-2/SopE (see tables \ref{tab:stm_phage} and \ref{tab:ty_phage}). With the exception of the SLP203 P22-like prophage in {\it S.} Typhimurium, all of these repressors lack the peptidase domain of the classical $\lambda$ repressor gene cI. This implies that the default anti-repression mechanism of Salmonella prophage may be more similar to a trans-acting mechanism recently discovered in Gifsy phage \parencite{Lemire2011} than to the $\lambda$ repressor's RecA-induced self-cleavage mechanism. We are also able to confirm that most phage remnants and fusions contained no active repressors, with the exception of the SLP281 degenerate P2-like prophage in {\it S.} Typhimurium. This degenerate prophage contains both intact replication and integration genes, but appears to lack tail and head proteins, suggesting it may depend on another phage for production of viral particles. Both genomes also encode P4-like satellite prophage, which rely on `helper' phage for lytic functions and utilize a complex antisense-RNA based regulation mechanism for decision pathways regarding cell fate \parencite{Briani2001} using structural homologs of the IsrK \parencite{Padalon-Brauch2008} and C4 ncRNAs \parencite{Forti2002}, known as seqA and CI RNA in the P4 literature, respectively. While the mechanism of P4 lysogenic maintenance is not known, the IsrK-like ncRNAs of two potentially active P4-like prophage in {\it S.} Typhi are required under TraDIS. This sequence element has previously been shown to be essential for the establishment of the P4 lysogenic state \parencite{Sabbattini1995}, and we predict based on our observations that it may be necessary for lysogenic maintenance as well. The fact that some lambdoid prophage in {\it S.} Typhimurium encode non-coding genes structurally similar to the IsrK-C4 immunity system of P4 raises the possibility that these systems may be acting as a defense mechanism of sorts, protecting the prophage from predatory satellite phage capable of co-opting its lytic genes.

% Table generated by Excel2LaTeX from sheet 'Sheet1'
%
\begin{table}
   \tiny
   \centering
   \noindent
    \caption[Phage elements in \textit{S.} Typhi]{\textbf{Phage elements in \emph{S.} Typhi.} Genomic coordinates determined from \textcite{Thomson2004} and manual inspection. Repressor domains and architecture were determined using the HMMER webserver \parencite{Finn2011} and Pfam \parencite{Punta2012}. Phage types were determined using repressor sequence similarity searches and information from \textcite{Thomson2004} and \textcite{Kropinski2007}. }
    \begin{tabular}{     m{0.5in}
    				m{0.4in}
				m{0.4in}
				m{0.6in}
				m{1.6in}
				m{0.4in}
				m{0.5in}
				m{0.5in}
				}
   
    \\
     \toprule
    \textbf{Element name} & \textbf{Genomic coordinates} & \textbf{Repressor} & \textbf{Repressor domain(s)} & \textbf{Repressor domain architecture} & \textbf{Predicted active?} & \textbf{Phage type} & \textbf{Required cargo} \\
    \midrule
    ST15  & 1408790 - 1441377 & N/A   & N/A   & N/A   & No    & Mu/P2 fusion & N/A \\
    Gifsy-2 & 1929572 - 1972330 & t1920 & HTH\_3 (PF01381) &    \includegraphics[height=6mm]{rep6}   & Yes   & lambdoid & N/A \\
    ST2-27 & 2735054 - 2745321 & IsrK RNA (RF01394) & N/A   & N/A   & Yes   & P4-like & N/A \\
    ST27  & 2745477 - 2768221 & N/A   & N/A   & N/A   & No    & P2/iroA fusion & N/A \\
    ST35  & 3500854 - 3536047 & t3402 & Phage\_CI\_repr (PF07022) &   \includegraphics[height=6mm]{rep7}    & Yes   & P2-like & t3415 \\
    SopE  & 4457346 - 4491316 & t4337 & Phage\_CI\_repr (PF07022) &   \includegraphics[height=6mm]{rep8}    & Yes   & P2-like & N/A \\
    N/A   & 4519423 - 4519501 & IsrK RNA (RF01394) & N/A   & N/A   & No    & remnant & N/A \\
    ST46  & 4666579 - 4677433 & IsrK RNA (RF01394) & N/A   & N/A   & Yes   & P4-like & N/A \\
    \bottomrule
    \end{tabular}%
    \label{tab:ty_phage}%
\end{table}



In addition to repressors, 4 prophage cargo genes in {\it S.} Typhimurium and one in {\it S.} Typhi are required (See tables \ref{tab:stm_phage}, \ref{tab:ty_phage}, \ref{tab:stm_uniq}, and \ref{tab:ty_uniq}). The {\it S.} Typhimurium prophage cargo genes encode a PhoPQ regulated protein, a protein predicted to be involved in natural transformation, an endodeoxyribonuclease, and a hypothetical protein. The {\it S.} Typhi prophage cargo gene encodes a protein containing the DNA-binding HIRAN\nomenclature[Z]{HIRAN}{HIP116, Rad5p N-terminal} domain \parencite{Iyer2006}, believed to be involved in the repair of damaged DNA. These warrant further investigation, as they are genes that have been recently acquired and become necessary for survival in rich media.

To compare differences between requirements for orthologous genes in both serovars, we calculated log-fold read ratios to eliminate genes which were classified differently in {\it S.} Typhi and {\it S.} Typhimurium but did not have significantly different read densities (see Methods.) Even after this correction, 36 {\it S.} Typhimurium genes had a significantly lower frequency of transposon insertion compared to the equivalent genes in {\it S.} Typhi (P $<$ 0.05), including four encoding hypothetical proteins (table \ref{tab:stm_uniq}). This indicates that these gene products play a vital role in {\it S.} Typhimurium but not in {\it S.} Typhi when grown under laboratory conditions. 

% Table generated by Excel2LaTeX from sheet 'Sheet1'
%
\begingroup
\begin{landscape}
   \tiny
   \noindent
    \begin{longtable}{ r
    				r
				r
				r
				r
				l
				r
				r
				r
				c
				p{1.8in}}
    \caption[Genes uniquely required in S. Typhimurium]{\textbf{Genes uniquely required in S. Typhimurium.} Genes determined to be uniquely required in S. Typhimurium. SL, S. Typhimurium; Ty, S. Typhi; inserts refer to the number of unique insertion sites within a gene; reads refer to the number of sequence reads over all insertions sites within a gene. �, P-value (associated with log2 read ratio) < 0.05. �, sseJ is a pseudogene in S. Typhi. Shaded rows indicate genes shown to be H-NS repressed in \textcite{Navarre2006}}
    \\
       \toprule
   & \textbf{Ty inserts} & \textbf{Ty reads} & \textbf{SL inserts} & \textbf{SL reads} & \textbf{SL ID} & \textbf{SL gene length} & \textbf{Ty ID} & \textbf{Ty gene length} & \textbf{Name} & \textbf{Function} \\
    \midrule
\multirow{22}{*}{\begin{sideways}\textbf{No ortholog in {\it S.} Typhi}\end{sideways}}   & -     & -     & 18    & 123   & SL0742 & 1269  & -     & -     & -     & putative cation transporter \\
   & -     & -     & 9     & 80    & SL0830 & 516   & -     & -     & -     & conserved hypothetical protein \\
   & -     & -     & 4     & 21    & SL0831 & 855   & -     & -     & -     & putative electron transfer flavoprotein (beta subunit) \\
   & -     & -     & 0     & 0     & SL0950 & 323   & -     & -     & -     & predicted bacteriophage protein, potential phage repressor Gifsy-2 \\
   & -     & -     & 11    & 75    & SL1179 & 789   & -     & -     & envF  & lipoprotein \\
   & -     & -     & 3     & 18    & SL1480 & 249   & -     & -     & -     & antitoxin Phd\_YefM, type II toxin-antitoxin system \\
   & -     & -     & 4     & 32    & SL1527 & 264   & -     & -     & ydcX  & putative inner membrane protein \\
   & -     & -     & 1     & 3     & SL1560 & 717   & -     & -     & -     & putative membrane protein \\
   & -     & -     & 7     & 50    & SL1601 & 859   & -     & -     & -     & putative transcriptional regulator (pseudogene) \\
   & -     & -     & 4     & 36    & SL1799 & 201   & -     & -     & -     & bacteriophage encoded pagK (phoPQ-activated protein) \\
   & -     & -     & 5     & 22    & SL1830A & 434   & -     & -     & -     & conserved hypothetical protein (pseudogene) \\
   & -     & -     & 3     & 27    & SL1967 & 677   & -     & -     & -     & predicted bacteriophage protein, potential phage repressor SLP203 \\
   & -     & -     & 1     & 15    & SL2045A & 63    & -     & -     & yoeI  & short ORF \\
   & -     & -     & 17    & 107   & SL2066 & 900   & -     & -     & rfbJ  & CDP-abequose synthase \\
   & -     & -     & 3     & 34    & SL2549 & 209   & -     & -     & -     & endodeoxyribonuclease \\
   & -     & -     & 4     & 149   & SL2593 & 449   & -     & -     & -     & putative DNA-binding protein, potential phage repressor Gifsy-1 SLP272 \\
   & -     & -     & 3     & 7     & SL2633 & 846   & -     & -     & -     & putative repressor protein, phage SLP281 \\
   & -     & -     & 2     & 21    & SL2695 & 978   & -     & -     & smf   & putative competence protein \\
   & -     & -     & 5     & 39    & SL4132 & 291   & -     & -     & -     & hypothetical protein \\
   & -     & -     & 5     & 45    & SL4354A & 303   & -     & -     & -     & conserved hypothetical protein \\
   \midrule
  \multirow{11}{*}{\begin{sideways}\parbox{2in}{\centering\textbf{Present in {\it S.} Typhi but required only in {\it S.} Typhimurium}}\end{sideways}} & 36    & 474   & 5     & 26    & SL0032 & 441   & t0033 & 306   & -     & putative transcriptional regulator \\
   & 71    & 349   & 11    & 48    & SL0623 & 642   & t2232 & 576   & lipB  & lipoate-protein ligase B \\
   & 151   & 3546  & 10    & 64    & SL0702 & 897   & t2156 & 894   & -     & putative glycosyl transferase \\
   & 194   & 3007  & 9     & 61    & SL0703 & 1134  & t2155 & 1134  & -     & galactosyltransferase \\
   & 231   & 3499  & 15    & 67    & SL0706 & 1779  & t2152 & 1780  & -     & putative glycosyltransferase, cell wall biogenesis \\
   & 84    & 1041  & 2     & 4     & SL0707 & 834   & t2151 & 834   & -     & putative glycosyltransferase, cell wall biogenesis \\
   & 49    & 367   & 14    & 70    & SL0722 & 1569  & t2136 & 1569  & cydA  & cytochrome d ubiquinol oxidase subunit I \\
   & 74    & 1613  & 5     & 22    & SL1069 & 693   & t1789 & 693   & -     & putative secreted protein \\
   & 20    & 199   & 1     & 1     & SL1203 & 150   & t1146 & 156   & -     & hypothetical protein \\
   & 20    & 290   & 1     & 5     & SL1264 & 315   & t1209 & 315   & -     & putative membrane protein \\
   & 84    & 384   & 6     & 26    & SL1327 & 402   & t1261 & 384   & spiC  & putative pathogenicity island 2 secreted effector protein \\
    \multirow{25}{*}{\begin{sideways}\parbox{3in}{\centering\textbf{Present in {\it S.} Typhi but required \\only in {\it S.} Typhimurium}}\end{sideways}}& 66    & 769   & 5     & 35    & SL1331 & 270   & t1265 & 327   & sseA  & T3SS chaperone \\
   & 36    & 307   & 2     & 5     & SL1341 & 228   & t1275 & 228   & ssaH  & putative pathogenicity island protein \\
   & 47    & 407   & 1     & 3     & SL1342 & 249   & t1276 & 249   & ssaI  & putative pathogenicity island protein \\
   & 144   & 3197  & 5     & 14    & SL1343 & 750   & t1277 & 750   & ssaJ  & putative pathogenicity island lipoprotein \\
   & 63    & 847   & 5     & 26    & SL1354 & 267   & t1288 & 267   & ssaS  & putative type III secretion protein \\
   & 73    & 762   & 4     & 44    & SL1355 & 780   & t1289 & 780   & ssaT  & putative type III secretion protein \\
   & 30    & 226   & 12    & 48    & SL1386 & 693   & t1322 & 693   & rnfE/ydgQ & Electron transport complex protein rnfE \\
   & 265   & 3337  & 29    & 165   & SL1473 & 1557  & t1463 & 1557  & pqaA  & PhoPQ-activated protein \\
   & 85    & 765   & 6     & 35    & SL1532 & 951   & t1511 & 951   & sifB  & putative virulence effector protein \\
   & 22    & 156   & 16    & 174   & SL1561 & 1227  & t1534� & 141   & sseJ  & Salmonella translocated effector protein (SseJ) \\
   & 119   & 1639  & 10    & 44    & SL1563 & 762   & t1536 & 762   & -     & putative periplasmic amino acid-binding protein \\
   & 107   & 2440  & 5     & 44    & SL1564 & 648   & t1537 & 648   & -     & putative ABC amino acid transporter permease \\
   & 183   & 1646  & 20    & 118   & SL1628 & 1355  & t1612 & 1364  & -     & hypothetical protein \\
   & 23    & 177   & 1     & 5     & SL1659 & 183   & t1640 & 183   & yciG  & conserved hypothetical protein \\
   & 78    & 617   & 16    & 104   & SL1684 & 1014  & t1664 & 1014  & hnr   & putative regulatory protein \\
   & 37    & 277   & 4     & 25    & SL1785 & 396   & t1022 & 396   & -     & conserved hypothetical protein \\
   & 166   & 2823  & 9     & 27    & SL1793 & 915   & t1016 & 915   & pagO  & inner membrane protein \\
   & 28    & 311   & 3     & 22    & SL1794 & 159   & t1015 & 159   & -     & putative inner membrane protein \\
   & 23    & 155   & 1     & 4     & SL1823 & 972   & t0988 & 972   & msbB  & lipid A acyltransferase \\
   & 60    & 402   & 11    & 58    & SL2064 & 1002  & t0786 & 1002  & rfbV  & putative glycosyl transferase \\
   & 87    & 524   & 7     & 59    & SL2065 & 1293  & t0785 & 1299  & rfbX  & putative O-antigen transporter \\
   & 66    & 559   & 13    & 74    & SL2069 & 774   & t0780 & 774   & rfbF  & glucose-1-phosphate cytidylyltransferase \\
   & 41    & 204   & 5     & 14    & SL3828 & 1830  & t3658 & 1830  & glmS  & glucosamine-fructose-6-phosphate aminotransferase \\
   & 27    & 288   & 5     & 23    & SL4250 & 288   & t4220 & 288   & -     & putative GerE family regulatory protein \\
   & 148   & 2633  & 16    & 89    & SL4251 & 876   & t4221 & 876   & -     & araC family regulatory protein \\
    \bottomrule
    \label{tab:stm_uniq}%
    \end{longtable}%
\end{landscape}%
\endgroup



\begin{figure}[htp]
\begin{center}
\includegraphics[width=14cm]{cell_surface}
\caption[Comparison of cell surface operon structure and requirements]{\textbf{Comparison of cell surface operon structure and requirements.} Diagram illustrating cell surface operons with different requirement patterns in {\it S.} Typhimurium and {\it S.} Typhi. The top figure is of an uncharacterized operon putatively involved in cell wall biogenesis, while the bottom figure shows a portion of the rfb operon involved in O-antigen biosynthesis. Plots along the top and bottom of each figure show insertions in {\it S.} Typhimurium and {\it S.} Typhi, respectively, with read depth on the y-axis with a maximum cut-off of 100 reads. Genes in blue are required in {\it S.} Typhimurium, genes in white are pseudogenes; others are in grey. Grey rectangles represent BLAST hits between orthologous genes, with percent nucleotide identity colored on the scale to the right of each figure.
} 
\label{fig:cell_surface}
\end{center}
\end{figure}

A major difference between the two serovars is in the requirement for genes involved in cell wall biosynthesis (see figure \ref{fig:cell_surface}). A set of four genes (SL0702, SL0703, SL0706, and SL0707) in an operonic structure putatively involved in cell wall biogenesis is required in {\it S.} Typhimurium but not in {\it S.} Typhi. The protein encoded by SL0706 is a pseudogene in {\it S.} Typhi (Ty2 unique ID: t2152) due to a 1bp deletion at codon 62 that causes a frameshift (Figure 4a). This operon contains an additional two pseudogenes in {\it S.} Typhi (t2154 and t2150), as well as a single different pseudogene (SL0700) in {\it S.} Typhimurium, indicating that this difference in gene requirements reflects the evolutionary adaptation of these serovars to their respective niches. Similarly, four genes ({\it rfbV}, {\it rfbX}, {\it rfbJ} and {\it rfbF}) within an O-antigen biosynthetic operon are required by {\it S.} Typhimurium but not {\it S.} Typhi. There appears to have been a shuffling of O-antigen biosynthetic genes since the divergence between the two serovars, and {\it rfbJ}, encoding a CDP-abequose\nomenclature[Z]{CDP}{Cytidine diphosphate glucose} synthase, has been lost from {\it S.} Typhi altogether. These broader requirements for cell wall-associated biosynthetic and transporter genes suggest that surface structure biogenesis is of greater importance in {\it S.} Typhimurium.

We also identified seven genes from the shared pathogenicity island SPI-2 that appear to contain few or no transposon insertions only in {\it S.} Typhimurium under laboratory conditions. These genes ({\it spiC}, {\it sseA}, and {\it ssaHIJT}) are thought to encode components of the SPI-2 type III secretion system apparatus (T3SS)\nomenclature[Z]{T3SS}{Type III secretion system} \parencite{Kuhle2004}. In addition, the effector genes {\it sseJ} and {\it sifB}, whose products are secreted through the SPI-2-encoded T3SS \parencite{Miao2000, Freeman2003}, also fell into the `required' category in {\it S.} Typhimurium alone. All of these genes display high A+T nucleotide sequence and have been previously shown (in {\it S.} Typhimurium) to be strongly bound by the nucleoid associated protein H-NS, encoded by {\it hns} \parencite{Lucchini2006, Navarre2006}. Therefore, rather than being `required', it is instead possible that access for the transposon was sufficiently restricted that very few insertions occurred at these sites. In further support of this hypothesis, a comparison of the binding pattern of H-NS detected in studies using {\it S.} Typhimurium LT2 with the TraDIS results from the SPI-2 locus indicated that high regions of H-NS enrichment correlated well with both the {\it ssa} genes described here and with {\it sseJ} (see figure \ref{fig:hns}). An earlier study also suggests that high-density DNA binding proteins can block Mu, Tn5, and Tn10 insertion \parencite{Manna2007}; however, a genome-wide study of the effects of H-NS binding on transposition would be necessary to confirm this effect.

\begin{figure}[htp]
\begin{center}
\includegraphics[width=14cm]{hns}
\caption[H-NS enrichment across the SPI-2 locus]{\textbf{H-NS enrichment across the SPI-2 locus.} Based on data from \textcite{Lucchini2006} where a 2 fold enrichment of H-NS-bound DNA over a total genomic DNA control in a ChIP-on-chip experiment was taken to indicate regions of H-NS binding in {\it S.} Typhimurium strain LT2. Assuming these binding patterns are similar in the {\it S.} Typhimurium strain tested in this study, H-NS binding may have affected transposon access to genes in the SPI-2 locus. 
} 
\label{fig:hns}
\end{center}
\end{figure}

% Table generated by Excel2LaTeX from sheet 'Sheet1'
%
\begin{table}
   \tiny
   \centering
   \noindent
    \caption[Candidate required genes affected by H-NS binding Typhimurium]{\textbf{Candidate required genes affected by H-NS binding.} Genes identified by comparison with data from \textcite{Navarre2006}. Fold change values, also from \textcite{Navarre2006}, indicating genes whose expression are significantly repressed by H-NS. }
    \begin{tabular}{ c
    				c
				c
				r
				l
				}
   
    \\
     \toprule
    \textbf{Gene} & \textbf{SL ID} & \textbf{STM ID} & \textbf{Fold change} & \textbf{Function} \\
    \midrule
    -     & SL0830 & STM0854 & -16.2 & conserved hypothetical protein \\
    -     & SL0831 & STM0855 & -33.8 & putative putative electron transfer flavoprotein (beta subunit) \\
    -     & SL1069 & STM1131 & -13.5 & putative putative secreted protein \\
    spiC  & SL1327 & STM1393 & -19.1 & putative pathogenicity island 2 secreted effector protein \\
    sseA  & SL1331 & STM1397 & -46   & Type three secretion system chaperone \\
    ssaH  & SL1341 & STM1407 & -8.8  & Type three secretion system apparatus \\
    ssaI  & SL1342 & STM1408 & -32.4 & putative putative pathogenicity island protein \\
    ssaJ  & SL1343 & STM1409 & -53.7 & putative putative pathogenicity island lipoprotein \\
            ssaS & SL1354 & STM1420 & -15.5 & putative putative type III secretion protein \\
    ssaT  & SL1355 & STM1421 & -33.9 & putative putative type III secretion protein \\
    pqaA  & SL1473 & STM1544 & -5.5  & PhoPQ-activated protein \\
    sifB  & SL1532 & STM1602 & -66.8 & putative putative virulence effector protein \\
    -     & SL1560 & STM1630 & -9.8  & putative putative membrane protein \\
    sseJ  & SL1561 & STM1631 & -48.6 & salmonella translocated effector protein (SseJ) \\
    -     & SL1563 & STM1633 & -91.9 & putative putative periplasmic amino acid-binding protein \\
    -     & SL1564 & STM1634 & -22.5 & putative putative ABC amino acid transporter permease \\
    -     & SL1628 & STM1698 & -101.4 & hypothetical protein \\
    -     & SL1659 & STM1728 & -17.3 & cytochrome b561 (cytochrome b-561) \\
    -     & SL1785 & STM1856 & -12.1 & conserved hypothetical protein \\
    pagO  & SL1793 & STM1862 & -11.9 & inner membrane protein (PagO) \\
    -     & SL1794 & STM1864 & -22.9 & putative inner membrane protein \\
    \bottomrule
    \end{tabular}%
    \label{tab:core}%
\end{table}



Indeed, the generation of null {\it S.} Typhimurium mutants in {\it sseJ} and {\it sifB}, as well as many others generated at the SPI-2 locus suggest that these genes are not truly a requirement for growth in this serovar \parencite{Freeman2003,Hensel1997,Hensel1998,Ohlson2005}. While this is a reminder that the interpretation of gene requirement needs to be made with care, the effect of H-NS upon transposon insertion is not genome-wide. If this were the case, there would be an under-representation of transposon mutants in high A+T regions (known for H-NS binding), which is not what we observed. In total, only 21 required genes fall into the `{\it hns}-repressed' category described in \textcite{Navarre2006}(see table \ref{tab:hns}); the remainder (almost 400) contained sufficient transposon insertions to conclude they were non-required. In addition, we noted that all SPI-1 genes that encode another T3SS and are of high A+T content were also found to be non-required. This phenomenon was not observed in {\it S.} Typhi, possibly because the strain used harbors the pHCM1 plasmid, which encodes the H-NS-like protein {\it sfh} and has been shown to affect H-NS binding \parencite{Doyle2007,Dillon2010}.

Twenty-two {\it S.} Typhi genes had a significantly lower frequency of transposon insertion compared to orthologs in {\it S.} Typhimurium (P $<$ 0.05), indicating that they are required only in {\it S.} Typhi for growth under laboratory conditions (table \ref{tab:ty_uniq}), including the {\it fepBDGC} operon. This indicates a requirement for ferric (Fe(III)\nomenclature{Fe(III)}{Ferric iron}) rather than ferrous (Fe(II)\nomenclature{Fe(II)}{Ferrous iron}) iron. This can be explained by the presence of Fe(III) in the bloodstream, where {\it S.} Typhi can be found during typhoid fever \parencite{Wain1998}. These genes function to recover the ferric chelator enterobactin from the periplasm, acting with a number of proteins known to aid the passage of this siderophore through the outer membrane \parencite{Rabsch1999}. It has long been noted that {\it aroA} mutants of {\it S.} Typhi, deficient in their ability to synthesize enterobactin, exhibit severe growth defects on complex media, while similar mutants of {\it S.} Typhimurium grow normally under the same conditions \parencite{Edwards1988}, though the mechanism has not been clear. Our results suggest that this difference in growth of {\it aroA} mutants is caused by a requirement for iron uptake through the {\it fep} system in {\it S.} Typhi. During host adaptation, {\it S.} Typhi has accumulated pseudogenes in many iron transport and response systems \parencite{McClelland2004}, presumably because they are not necessary for survival in the niche {\it S.} Typhi occupies in the human host, which may have led to this dependence on {\it fep} genes. In contrast, {\it S.} Typhimurium generally causes intestinal rather than systemic infection and is able to utilize a wider range of iron sources, including Fe(II), a soluble form of iron present under anaerobic conditions such as those found in the intestine \parencite{Tsolis1996}. 

% Table generated by Excel2LaTeX from sheet 'Sheet1'
%
\definecolor{Gray}{gray}{0.9}
\begingroup
\begin{landscape}
\begin{table}
   \tiny
   \noindent
    \caption[Genes uniquely required in {\it S.} Typhi]{\textbf{Genes uniquely required in {\it S.} Typhi.} Genes determined to be uniquely required in {\it S.} Typhi. SL, {\it S.} Typhimurium; Ty, {\it S.} Typhi; inserts refer to the number of unique insertion sites within a gene; reads refer to the number of sequence reads over all insertions sites within a gene.  \textdagger, P-value (associated with log2 read ratio) $<$ 0.05. *, the assignment of recA as a required gene has been described previously \parencite{Langridge2009a}, but briefly is believed to be due to the presence of the priC pseudogene in Typhi.}
        \begin{tabular}{ r
    				r
				r
				r
				r
				l
				r
				l
				r
				c
				p{1.8in}}
    \\
        \toprule
   & \textbf{SL inserts} & \textbf{SL reads} & \textbf{Ty inserts} & \textbf{Ty reads} & \textbf{Ty ID} & \textbf{Ty gene length} & \textbf{SL ID} & \textbf{SL gene length} & \textbf{Name} & \textbf{Function} \\
    \midrule
  \multirow{7}{*}{\begin{sideways}\parbox{1in}{\centering\textbf{No ortholog in {\it S.} Typhimurium}}\end{sideways}}& -     & -     & 1     & 5     & t1332 & 132   & -     & -     & malY  & pseudogene \\
  & -     & -     & 2     & 32    & t1920 & 405   & -     & -     & -     & possible repressor protein, prophage 10/Gifsy-2 \\
  & -     & -     & 2     & 12    & t3157 & 165   & -     & -     & -     & conserved hypothetical protein \\
  & -     & -     & 2     & 12    & t3166 & 228   & -     & -     & -     & spurious ORF annotation overlapping the RnaseP/M1 RNA \\
  & -     & -     & 6     & 196   & t3402 & 570   & -     & -     & cI    & repressor protein, cs 73 prophage \\
  & -     & -     & 4     & 58    & t3415 & 741   & -     & -     & -     & HIRAN-domain family gene, potential DNA repair \\
  & -     & -     & 1     & 6     & t4531 & 150   & -     & -     & -     & hypothetical secreted protein \\
    \midrule
    \multirow{22}{*}{\begin{sideways}\parbox{2.8in}{\centering\textbf{Present in {\it S.} Typhimurium but required only in {\it S.} Typhi}\textsuperscript{\textdagger}}\end{sideways}} &199   & 1792  & 18    & 59    & t0095 & 1287  & SL0093 & 1287  & surA  & survival protein SurA precursor \\
    &45    & 498   & 3     & 22    & t0123 & 459   & SL0119 & 459   & yabB/mraZ & conserved hypothetical protein \\
    &120   & 589   & 11    & 32    & t0203 & 1281  & SL0203 & 1281  & hemL  & glutamate-1-semialdehyde 2,1-aminomutase \\
    &123   & 982   & 2     & 25    & t0224 & 1353  & SL0224 & 1353  & yaeL/rseP & Zinc metallopeptidase \\
    &67    & 452   & 1     & 14    & t0270 & 576   & SL2604 & 576   & rpoE  & RNA polymerase sigma-E factor \\
    &140   & 760   & 0     & 0     & t0587 & 2286  & SL2246 & 2286  & nrdA  & ribonucleoside-diphosphate reductase 1 alpha chain \\
    &113   & 641   & 15    & 42    & t2140 & 2802  & SL0718 & 2802  & sucA  & 2-oxoglutarate dehydrogenase E1 component \\
    &116   & 753   & 13    & 36    & t2177 & 1641  & SL0680 & 1641  & pgm   & phosphoglucomutase \\
    &80    & 542   & 9     & 15    & t2276 & 1008  & SL0580 & 1008  & fepD  & ferric enterobactin transport protein FepD \\
    &93    & 591   & 2     & 2     & t2277 & 990   & SL0579 & 990   & fepG  & ferric enterobactin transport protein FepG \\
    &64    & 508   & 5     & 6     & t2278 & 795   & SL0578 & 795   & fepC  & ferric enterobactin transport ATP-binding protein FepC \\
    &201   & 1129  & 12    & 116   & t2410 & 2355  & SL0444 & 2355  & lon   & Lon protease \\
    &95    & 518   & 8     & 20    & t2730 & 1062  & SL2809 & 1062  & recA* & recA protein \\
    &135   & 719   & 16    & 39    & t2996 & 1992  & SL3052 & 1947  & tktA  & transketolase \\
    &76    & 358   & 3     & 9     & t3120 & 1434  & SL3173 & 1434  & rfaE  & ADP-heptose synthase \\
    &213   & 1976  & 6     & 50    & t3265 & 1071  & SL3321 & 1071  & degS  & serine protease \\
    &43    & 448   & 3     & 10    & t3326 & 606   & SL3925 & 606   & yigP  & conserved hypothetical protein \\
    &124   & 571   & 17    & 36    & t3384 & 2025  & SL3872 & 2025  & rep   & ATP-dependent DNA helicase \\
    &175   & 1208  & 6     & 21    & t3621 & 2787  & SL3947 & 2787  & polA  & DNA polymerase I \\
    &117   & 797   & 9     & 13    & t3808 & 1047  & SL3677 & 1047  & waaF  & ADP-heptose-LPS heptosyltransferase II \\
    &176   & 1628  & 14    & 59    & t4153 & 1080  & SL4183 & 1080  & alr   & alanine racemase \\
    &140   & 1127  & 10    & 38    & t4411 & 951   & SL4294 & 951   & miaA  & tRNA delta-2-isopentenylpyrophosphate transferase \\
    \bottomrule
   \end{tabular}
    \label{tab:ty_uniq}%
    \end{table}%
\end{landscape}%
\endgroup



\subsection{TraDIS provides resolution sufficient to evaluate ncRNA contributions to fitness}
Under a Poisson approximation to the transposon insertion process, a region of 41 (in {\it S.} Typhimurium) or 60 bases (in {\it S.} Typhi) has only a 1\% probability of not containing an insertion. NcRNAs tend to be considerably shorter than their protein-coding counterparts, but this gives us sufficient resolution to assay most of the non-coding complement of the {\it Salmonella} genome. As a proof of principle, we performed an analysis of the best-understood class of small ncRNAs, the tRNAs. Francis Crick hypothesized that a single tRNA could recognize more than one codon through wobble recognition \parencite{Crick1966}, where a non-canonical G-U base pair is formed between the first (wobble) position of the anticodon and the third nucleotide in the codon. As a result, some codons are covered by multiple tRNAs, while others are covered non-redundantly by a single tRNA. We expect that singleton wobble-capable tRNAs, that is wobble tRNAs which recognize a codon uniquely, will be required. In addition, we inferred the requirement for other tRNAs through the non-redundant coverage of their codons and used this to benchmark our ability to use TraDIS to reliably interrogate short genomic intervals.

The {\it S.} Typhi and {\it S.} Typhimurium genomes encode 78 and 85 (plus one pseudogene) tRNAs respectively with 40 anticodons, as identified by tRNAscan-SE \parencite{Lowe1997}. In {\it S.} Typhi, 10 out of 11 singleton wobble tRNAs are predicted to be required or ambiguous, compared to 16 tRNAs below the ambiguous LLR cut-off overall (significant enrichment at the 0.05 level, two-tailed Fisher�s exact test p-value: 6.4e-08.) Similarly in {\it S.} Typhimurium, 9 of 11 singleton wobble tRNAs are required or ambiguous compared to 15 required or ambiguous tRNAs overall, again showing a significant enrichment of required tRNAs in this subset (Fisher�s exact test p-value: 5.2e-07.)  The one singleton wobble tRNA which is consistently not required in both serovars is the tRNA-Pro(GGG), which occurs within a 4-member codon family. It has previously been shown in {\it S.} Typhimurium that tRNA-Pro(UGG) can read all four proline codons in vivo due to a cmo5U34 modification to the anticodon, obviating the need for a functional tRNA-Pro(GGG) \parencite{Nasvall2004} and making this tRNA non-required. The other non-required singleton wobble tRNA in {\it S.} Typhimurium, tRNA-Leu(GAG), is similarly a member of a 4-member codon family. We predict tRNA-Leu(TAG) may also be capable of recognizing all 4 leucine codons in this serovar; such a leucine ``four-way wobble'' has been previously inferred in at least one other bacterial species \parencite{Osawa1992,Marck2002}.

Of the 6 required non-wobble tRNAs in each serovar, four are shared. These include two non-wobble singleton tRNAs covering codons uniquely, as well as a tRNA with the ATG anticodon which is post-transcriptionally modified by the required protein MesJ/TilS to recognize the isoleucine codon ATA \parencite{Marck2002}. An additional two required tRNAs in both serovars, one shared and one with a differing anticodon, contain Gln anticodons and are part of a polycistronic tRNA operon containing other required tRNAs. This operon is conserved in E. coli with the exception of an additional tRNA-Gln at the 3' end that has been lost in the Salmonella lineage. It is possible that transposon insertions early in the operon may interfere with processing of the polycistronic transcript in to mature tRNAs. Finally, we do not observe insertions in a tRNA-Met and a tRNA-Val in {\it S.} Typhi and {\it S.} Typhimurium, respectively.

Using this analysis of the tRNAs we estimate a worst-case PPV for these short molecules ($\sim$76 bases) at 81\%, in line with our previous estimates for conserved protein-coding genes, and a FPR of $<$4\%, higher than for protein-coding genes but still well within the typical tolerance of high-throughput experiments. This assumes that the ``required'' operonic tRNA-Glns and the serovar-specific tRNA-Met and tRNA-Val are all false positives; it is not clear that this is in fact the case.

Surveying the shared required ncRNA content of both serovars (see table \ref{tab:ncrna}), we find that the RNA components of the signal recognition particle (SRP)\nomenclature[Z]{SRP}{Signal recognition particle} and RNase P, two universally conserved ncRNAs, are required as expected. The SRP is an essential component of the cellular secretion machinery, while RNase P is necessary for the maturation of tRNAs. We also find a number of required known and potential cis-regulatory molecules associated with genes required for growth under laboratory conditions in both serovars. The FMN\nomenclature[Z]{FMN}{Flavin mononucleotide} riboswitch controls {\it ribB}, a 3,4-dihydroxy-2-butanone 4-phosphate synthase involved in riboflavin biosynthesis, in response to flavin mononucleotide concentrations \parencite{Winkler2002}. Additionally, we are able to assign putative functions to a number of previously uncharacterized required non-coding transcripts through their $5'$ association with required genes. SroE, a 90 nucleotide molecule discovered in an early sRNA screen \parencite{Vogel2003}, is consistently located at the $5'$ end of the required {\it hisS} gene across its phylogenic distribution in the Enterobacteriaceae. Given this consistent association and the function of HisS as a histidyl-tRNA synthetase, we hypothesize that this region may act in a manner similar to a T-box leader, inducing or repressing expression in response to tRNA-His levels. The {\it thrU} leader sequence, recently discovered in a deep-sequencing screen of E. coli \parencite{Raghavan2011}, appears to regulate a polycistronic operon of required singleton wobble tRNAs. Three additional required cis-regulatory elements, t44, S15, and StyR-8, are associated with required ribosomal proteins, highlighting the central role ncRNA elements play in regulating fundamental cellular processes. 
\newpage

% Table generated by Excel2LaTeX from sheet 'Sheet1'
%
\definecolor{Gray}{gray}{0.9}
\begingroup
   \tiny
   \noindent
    \begin{longtable}{ l
    				p{0.3in}
				p{1.5in}
				c
				p{0.5in}
				p{0.5in}
				p{1in}
				}
    \caption[Candidate required ncRNAs]{\textbf{Candidate required ncRNAs greater than 60 nucleotides in length, excluding rRNA and tRNA.} Known and putative non-coding elements classified as required or ambiguous in this screen. Required ncRNAs have a log-likelihood ratio (LLR) between required and non-required models of < -2; see Methods. * \textdagger, ncRNAs which are amibiguous (LLR between -2 and 2) in S. Typhi(*) or in S. Typhimurium(\textdagger). Hfq-binding annotations are taken from \textcite{Chao2012}. The downstream protein-coding genes columns report annotated CDS or ribosomal RNA start sites within 100 bases of each candidate required non-coding element on either strand, and whether these downstream sequences are also classified as required.}
    \\
     \toprule
    \textbf{Name} & \textbf{Rfam accession} & \textbf{Function} & \textbf{Hfq-binding} & \textbf{Downstream protein-coding gene(s)} & \textbf{Downstream gene required} & \textbf{References} \\
    \midrule
    \multicolumn{7}{l}{\textbf{Required or ambiguous in both {\it S.} Typhi and {\it S.} Typhimurium}}\\
    SRP   & RF00169 & RNA component of the signal recognition particle &       &       &       & \textcite{Rosenblad2009} \\
    RNase P & RF00010 & RNA component of RNase P &       & ybaZ  & N     & \textcite{Frank1998} \\
    RFN   & RF00050 & FMN-sensing riboswitch controlling the ribB gene &       & ribB  & Y     & \textcite{Winkler2002} \\
    SroE  & RF00371 & Putative cis-regulatory element controlling the hisS gene &       & hisS  & Y     & \textcite{Vogel2003} \\
    ThrU Leader & NA    & Putative cis-regulatory element controlling the ThrU tRNA operon &       &       &       & \textcite{Raghavan2011} \\
    t44   & RF00127 & Cis-regulatory element controlling the ribosomal rpsB gene &       & rpsB  & Y     & \textcite{Tjaden2002,Aseev2008,Meyer2009} \\
    S15\textsuperscript{\textdagger}  & RF00114 & Translational regulator of the ribosomal S15 protein &       & rpsO  & Y     & \textcite{Benard1996} \\
    StyR-8 & NA    & Putative cis-regulatory element controlling the ribosomal rpmB gene &       & rpmB  & Y     & \textcite{Chinni2010} \\
    MicA  & RF00078 & sRNA involved in cellular response to extracytoplasmic stress & Y     & luxS  & N     & \textcite{Vogel2009} \\
    DsrA\textsuperscript{\textdagger} & RF00014 & sRNA regulator of H-NS & Y     & mngB  & N     & \textcite{Lease1998} \\
    STnc10 & NA    & Putative sRNA &       & nhaA  & N     & \textcite{Sittka2008} \\
    STnc60\textsuperscript{\textdagger} & NA    & Putative sRNA &       &       &       & \textcite{Sittka2008} \\
    STnc840 & NA    & Verified sRNA derived from 3' UTR of the flgL gene & Y     &       &       & \textcite{Chao2012} \\
    IS0420*\textsuperscript{\textdagger} & NA    & Putative ncRNA &       & rmf   & N     & \textcite{Raghavan2011,Chen2002} \\
    RGO0\textsuperscript{\textdagger} & NA    & Putative sRNA identified in E. coli &       &       &       & \textcite{Raghavan2011} \\
     \multicolumn{7}{l}{\textbf{Required or ambiguous in {\it S.} Typhimurium only}}\\
    rne5  & RF00040 & RNase E autoregulatory 5' element  &       & rne   & Y     & \textcite{Diwa2000} \\
    RydC  & RF00505 & sRNA regulator of the yejABEF ABC transporter & Y     &       &       & \textcite{Antal2005} \\
    RydB  & RF00118 & Putative ncRNA &       &       &       & \textcite{Wassarman2001} \\
    STnc510 & NA    & Putative sRNA &       & pagD/pagC & Y/N   & \textcite{Sittka2008} \\
    STnc460\textsuperscript{\textdagger} & NA    & Putative sRNA &       &       &       & \textcite{Sittka2008} \\
    STnc170 & NA    & Putative sRNA &       & SL1458 & N     & \textcite{Sittka2008} \\
    STnc130 & NA    & Putative sRNA &       & dmsA  & N     & \textcite{Sittka2008} \\
    RseX  & RF01401 & sRNA regulator of OmpA and OmpC & Y     &       &       & \textcite{Douchin2006} \\
    IsrJ  & RF01393 & sRNA regulator of SPI-1 effector protein secretion &       &       &       &  \textcite{Sittka2008, Padalon-Brauch2008} \\
    IsrI  & RF01392 & Island-encoded Hfq-binding sRNA & Y     & SL1028 & Y     &  \textcite{Sittka2008, Padalon-Brauch2008, Chao2012} \\
     \multicolumn{7}{l}{\textbf{Required or ambiguous in {\it S.} Typhi only}}\\
    RybB  & RF00110 & sRNA involved in cellular response to extracytoplasmic stress & Y     &       &       &  \textcite{Vogel2009} \\
    tk5*  & NA    & Putative ncRNA &       &       &       & \textcite{Raghavan2011,Rivas2001} \\
    STnc750 & NA    & Verified sRNA & Y     & SpeB  & N     & \textcite{Kroger2012, Chao2012} \\
    StyR-44* & RF01830 & Putative multicopy (2/6 copies required in S. Typhi) ncRNA associated with ribosomal RNA operon &       & 23S rRNA & N     &  \textcite{Chinni2010} \\
    tp2   & NA    & Putative ncRNA &       & aceE  & N     &  \textcite{Raghavan2011,Rivas2001} \\
    RdlD  & RF01813 & RdlD RNA anti-toxin, 1/2 copies required in S. Typhi &       &       &       & \textcite{Kawano2002} \\
    STnc120* & NA    & Putative sRNA &       &       &       & \textcite{Sittka2008} \\
    tp28* & NA    & Putative ncRNA &       & fur   & N     &  \textcite{Raghavan2011,Rivas2001} \\
    Phe Leader* & RF01859 & Phenylalanine peptide leader sequence associated with the required PheST operon &       & PheS  & Y     & \textcite{Zurawski1978} \\
    RimP Leader & RF01770 & Putative cis-regulator of the rimP-nusA-infB operon &       & rimP  & Y     & \textcite{Naville2010} \\
    GlmY  & RF00128 & Trans-acting regulator of the GlmS gene &       &       &       & \textcite{Urban2008} \\
    \bottomrule
    \label{tab:ncrna}%
    \end{longtable}%
\endgroup



\subsection{sRNAs required for competitive growth}

Inferring functions for potential trans-acting ncRNA molecules, such as anti-sense binding small RNAs (sRNAs), from requirement patterns alone is more difficult than for cis-acting elements, as we cannot rely on adjacent genes to provide any information.  It is also important to keep in mind that TraDIS assays requirements after a brief competition within a large library of mutants on permissive media. This may be particularly important when surveying the bacterial sRNAs, which are known to participate in responses to stress \parencite{Vogel2009a}.  

\begin{figure}[htp]
\begin{center}
\includegraphics[width=14cm]{sRNAs}
\caption[Proposed differences in sRNA utilization]{\textbf{Proposed differences in sRNA utilization.} Diagram illustrating inferred required sRNA regulatory networks under TraDIS. Molecules required in {\it S.} Typhi are highlighted in yellow and in {\it S.} Typhimurium are highlighted in blue. RseA, in yellow/grey check, is ambiguous in {\it S.} Typhi. Non-required molecules are in grey. Diamonds indicate sRNAs, circles regulatory proteins, ovals proteases, oblong shapes are membrane-anchored proteins, and rounded squares are outer membrane porins.
} 
\label{fig:sRNAs}
\end{center}
\end{figure}

This is demonstrated by two sRNAs involved in the $\sigma^E$-mediated extracytoplamic\nomenclature[G]{$\sigma^E$}{$\sigma^{24}$, extracytoplasmic stress sigma factor} stress response, RybB and RseX, both of which can be successfully knocked out in {\it S.} Typhimurium (83). In {\it S.} Typhi, {\it rpoE} is required, as it also is in {\it E. coli} \parencite{Baba2006, DeLasPenas1997}. However, in {\it S.} Typhimurium, {\it rpoE} tolerates a heavy insertion load, implying that $\sigma^E$ mutants are not disadvantaged in competitive growth. In {\it S.} Typhimurium, the sRNA RseX is required. Overexpression of RseX has previously been shown to compensate for $\sigma^E$ essentiality in E. coli by leading to the degradation of {\it ompA} and {\it ompC} transcripts (85). This suggests that RseX may also be short-circuiting the $\sigma^E$ stress response network in {\it S.} Typhimurium (figure \ref{fig:sRNAs}). To our knowledge, this is the first evidence of a native (i.e. not experimentally induced) activity of RseX. 

{\it S.} Typhi on the other hand requires $\sigma^E$ along with its activating proteases RseP and DegS and anchoring protein RseA, as well as the $\sigma^E$-dependent sRNA RybB, which also regulates OmpA and OmpC in {\it S.} Typhimurium, along with a host of other OMPs\nomenclature[Z]{OMP}{Outer membrane protein} \parencite{Papenfort2006}. It is unclear why the $\sigma^E$ response is required in {\it S.} Typhi but not {\it S.} Typhimurium, though it may partially be due to the major differences in the cell wall and outer membrane between the two serovars. In addition, there are significant differences in the OMP content of the {\it S.} Typhi and {\it S.} Typhimurium membranes that may be driving alternative mechanisms for coping with membrane stress. For instance, {\it S.} Typhi completely lacks OmpD, a major component of the {\it S.} Typhimurium outer membrane \parencite{Santiviago2003} and a known target of RybB \parencite{Vogel2009a}. 

Two additional sRNAs involved in stress response are also required by both {\it S.} Typhi and {\it S.} Typhimurium. The first, MicA, is known to regulate {\it ompA} and the {\it lamB} porin-coding gene in {\it S.} Typhimurium \parencite{Bossi2007}, contributing to the extracytoplasmic stress response. The second, DsrA, has been shown to negatively regulate the nucleoid-forming protein H-NS and enhance translation of the stationary-phase alternative sigma factor $\sigma^S$\nomenclature[G]{$\sigma^S$}{$\sigma^{38}$, starvation/stationary phase sigma factor} in {\it E. coli} \parencite{Lease1998}, though its regulation of $\sigma^S$ does not appear to be conserved in {\it S.} Typhimurium \parencite{Jones2006}. Both have been previously deleted in {\it S.} Typhimurium, and so are not essential. H-NS knockouts have previously been shown to have severe growth defects in {\it S.} Typhimurium that can be rescued by compensatory mutations in either the {\it phoPQ} two-component system or {\it rpoS}, implying that the lack of H-NS is allowing normally silenced detrimental regions to be transcribed \parencite{Navarre2006}. As MicA has recently been shown to negatively regulate PhoPQ expression in E. coli \parencite{Coornaert2010}, it is tempting to speculate that MicA may be moderating the effects of DsrA-induced H-NS repression; however, it is currently unclear whether sRNA regulons are sufficiently conserved between {\it E. coli} and {\it S.} enterica to justify this hypothesis.

\section{Conclusions}

The extremely high resolution of TraDIS has allowed us to assay gene requirements in two very closely related salmonellae with different host ranges. We found, under laboratory conditions, that 58 genes present in both serovars were required in only one, suggesting that identical gene products do not necessarily have the same phenotypic effects in the two different serovar backgrounds. Many of these genes occur in genomic regions or metabolic systems which contain pseudogenes and/or have undergone reorganization since the divergence of {\it S.} Typhi and {\it S.} Typhimurium, demonstrating the complementarity of TraDIS and phylogenetic analysis. These changes may in part explain differences observed in the pathogenicity and host specificity of these two serovars. In particular, {\it S.} Typhimurium showed a requirement for cell surface structure biosynthesis genes; this may be partially explained by the fact that {\it S.} Typhi expresses the Vi-antigen which masks the cell surface, though these genes are not required for survival in our assay. {\it S.} Typhi on the other hand has a requirement for iron uptake through the {\it fep} system, which enables ferric enterobactin transport. This dependence on enterobactin suggests that {\it S.} Typhi is highly adapted to the iron-scarce environments it encounters during systemic infections. Furthermore, this appears to represent a single point of failure in the {\it S.} Typhi iron utilization pathways, and may present an attractive target for narrow-spectrum antibiotics. 

Of the approximately 4500 protein coding genes present in each serovar, only about 350 were sufficiently depleted in transposon insertions to be classified as required for growth in rich media. This means that over 92\% of the coding genome has sufficient insertion density to be queried in future assays. Dense transposon mutagenesis libraries have been used to assay gene requirements under conditions relevant for infection, including {\it S.} Typhi survival in bile \parencite{Langridge2009a}, {\it Mycobacterium tuberculosis} catabolism of cholesterol \parencite{Griffin2011}, drug resistance in {\it Pseudomonas aeruginosa} \parencite{Gallagher2011}, and {\it Haemophilus influenzae} survival in the lung \parencite{Gawronski2009}. We expect that parallel experiments querying gene requirements under the same conditions in both serovars examined in this study will yield further insights in to the differences in the infective process between Typhi and Typhimurium, and ultimately the pathways that underlie host-adaptation.

Both serovars possess substantial complements of horizontally-acquired DNA. We have been able to use TraDIS to assay these recently acquired sequences. In particular, we�ve been able to identify, on a chromosome wide scale, active prophage through the requirement for their repressors. The P4 phage utilizes an RNA-based system to make decisions regarding cell fate, and structurally similar systems are used by P1, P7, and N15 phage \parencite{Citron1990,Ravin1999}. C4-like transcripts have been regarded as the primary repressor of lytic functions, though the IsrK-like sequence is known to be essential to the establishment of lysogeny in P4 and is transcribed in at least two phage types \parencite{Sabbattini1995, Ravin1999}. Our observations in {\it S.} Typhi suggest an important role for the IsrK-like sequence in maintenance of the lysogenic state in P4-like phage, though the mechanism remains unclear.

Recent advances in high-throughput sequencing have greatly enhanced our ability to detect novel transcripts, such as ncRNAs and short open reading frames (sORFs). In fact, our ability to identify these transcripts now far out-strips our ability to experimentally characterize these sequences. There have been previous efforts at high-throughput characterization of bacterial sRNAs and sORFs in enteric bacteria; however, these have relied on labor-intensive directed knockout libraries \parencite{Santiviago2009, Hobbs2010}. Here we have demonstrated that TraDIS has sufficient resolution to reliably query genomic regions as short as 60 bases, in agreement with a recent high-throughput transposon mutagenesis study in the $\alpha$-proteobacteria {\it Caulobacter crescentus} \parencite{Christen2011}. Our method has the major advantage that library construction does not rely upon genome annotation, and newly discovered elements can be surveyed with no further laboratory work. 

We have been able to assign putative functions to a number of ncRNAs using TraDIS though consideration of their genomic and experimental context. In addition, ncRNA characterization generally is done in model organisms like {\it E. coli} or {\it S.} Typhimurium, and it is unclear how stable ncRNA regulatory networks are over evolutionary time. By assaying two serovars of {\it Salmonella} with the same method under the same conditions, we have seen hints that there may be differences in sRNA regulatory networks between {\it S.} Typhi and {\it S.} Typhimurium. In particular, we have found that under the same experimental conditions, {\it S.} Typhi appears to rely on the $\sigma^{E}$ stress response pathway while {\it S.} Typhimurium does not; it is tempting to speculate that this difference in stress response is mediated by the observed difference in requirement for two sRNAs, RybB and RseX. We believe that this combination of high-throughput transposon mutagenesis with a careful consideration of the systems context of individual genes provides a powerful tool for the generation of functional hypotheses. We anticipate that the construction of TraDIS libraries in additional organisms, as well as the passing of these libraries through relevant experimental conditions, will provide further insights into the function of bacterial ncRNAs in addition to the protein-coding gene complement. 
