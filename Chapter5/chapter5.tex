% \pagebreak[4]
% \hspace*{1cm}
% \pagebreak[4]
% \hspace*{1cm}
% \pagebreak[4]
%\usepackage[round,colon,authoryear]{natbib}

\chapter{Kingdom-wide discovery of bacterial intrinsic termination motifs}
\label{sec:chapterPingpong}
\ifpdf
    \graphicspath{{Chapter5/Chapter5Figs/EPS/}{Chapter5/Chapter5Figs/}}
\fi

\section{Introduction}

As discussed in the pervious chapter, intrinsic termination of transcription is a fundamental cellular process in many, if not all, bacterial species. As reviewed in the previous chapter, the bulk of work on intrinsic termination has focused on canonical Rho-independent terminators (RITs), consisting of a G/C-rich hairpin structure followed by a poly-U tail. This is due to both their prevalence in model organisms such as \textit{Escherichia coli} and \textit{Bacillus subtilis}, as well as the distinctiveness of this motif making it an easy target for automated classification.

Despite this focus on canonical RITs, a number of intrinsic terminators which do not rely on a poly-U tail for termination activity are known. These include synthetic constructs derived from canonical RITs \parencite{Abe1996}, as well as naturally occurring terminators in \textit{Streptomyces} \parencite{Deng1987, Neal1991, Ingham1995} and \textit{Mycobacteria} \parencite{Unniraman2001}. Additionally, a number of ncRNA screens in Actinobacteria have described potential non-canonical RITs terminating ncRNA transcription \parencite{Swiercz2008,Miotto2012, Li2013}.  However, a more wide-spread effort at characterization of these elements has been hampered by two factors: their occurrence primarily in non-model organisms such as the Actinobacteria, and a lack of a systematic classification of these elements making it difficult to determine how wide-spread such elements are. The only study surveying potential alternative intrinsic terminators in the bacterial kingdom relied primarily on categorizing elements based on the shape of their predicted secondary structure \parencite{Unniraman2002}. However, this fails to consider the large number of very different sequences that can give rise to any particular secondary structure \parencite{Schuster1994}. It is well understood from studies of synthetic perturbations of canonical RITs that the sequence of both the hairpin structure and flanking sequence can have large, and often unexpected, effects on termination efficiency \parencite{Reynolds1992, Abe1996, Cambray2013, Chen2013}; there is no reason to think that non-canonical RITs would not exhibit a similar pattern of sequence specificity. As a result, there is a need for a robust classification of potential non-canonical RITs which considers both the sequence and structural features of these elements so that they can be systematically investigated. 

%Our view of bacterial diversity is expanding rapidly, particularly through the targeted sequencing of under-explored regions of the phylogeny \parencite{Wu2009} and recent advances in single-cell genomics enabling the sequencing of uncultivated organisms \parencite{Marcy2007, Rinke2013}, 

In the previous chapter I showed that covariance models (CMs) are able to capture sequence as well as structural features of canonical as well as putative non-canonical RITs. In this chapter I describe a method for the discovery of potential structured termination motifs across the bacterial kingdom, present an initial analysis of the elements discovered, and provide evidence for their activity through the analysis of a large collection of publicly-available RNA-seq data.

\section{Methods}

\textit{James Hadfield (University of Canterbury) ran the MCL clustering under my supervision. Paul P. Gardner (University of Canterbury) developed and ran the analysis of expression data, and assisted in manual curation of cluster alignments. I performed all other work described here. }

\subsection{Genome-wise motif discovery}
1853 embl format files containing the genomic sequence and annotations for 1639 bacteria were obtained from the EMBL European Nucleotide Archive completed bacterial genomes pages, see supplementary information for organisms and accession numbers.

Each embl file was screened independently for putative multi-copy termination motifs. For each embl file, I extracted sequences from -20 to +80 around annotated ORF\nomenclature[Z]{ORF}{Open reading frame} stop site. Each extracted sequence was screened for a lower than expected predicted MFE using RNAfold in order to screen out locally GC-rich but unstructured sequences. The sequence under consideration was shuffled 1000 times preserving dinucleotide frequencies, and a Gumbel distribution was fitted to the resulting empirical null MFE distribution using the R MASS package \parencite{Venables1994}. Sequences with a native MFE below the 95th percentile of the null distribution were discarded. The resulting set of sequences was then given as input to CMfinder \parencite{Yao2006}, which produces collections of locally-aligned structurally conserved motifs. I built a CM for each motif using Infernal 1.0.2 \parencite{Nawrocki2009}.  The resulting CMs were searched against the embl file the motif was discovered in, and were then screened on the following criteria for the collection of search hits with an E-value of less than 1: a copy number of between 100 and 3000, and a median distance of $<$10 to the nearest annotated ORF stop site. This resulted in a collection of 4359 putative termination motif CMs, each derived from a single embl file. 

\subsection{Clustering covariance models}

In order to cluster CMs, I developed an extension of MCL-based clustering \parencite{Enright2002} to generative models of sequence variation. I call the measure of CM similarity I developed for this purpose the reciprocal similarity score (RSS), defined as: \[ \left[\frac{\sum_{i=1}^{1000} -\ln{(E_{n,m,i})} +  \sum_{j=1}^{1000} -\ln{(E_{m,n,j})}}{2000}\right] + \ln(1000) \] where $E_{n,m,i}$ is the E-value of the $i$th sequence emitted by model $n$ scored by model $m$. Briefly, for each pair of CMs 1000 sequences were emitted from each CM and reciprocally scored with the other CM. The average of the negative log-transformed E-values was calculated, then shifted to be strictly positive by adding $\ln(1000)$ to generate the RSS\nomenclature[Z]{RSS}{Reciprocal similarity score} appropriate for use with MCL. MCL was run over the resulting RSS matrix, and the 100 largest clusters, ranging in size from 332 to 6 CMs, were taken forward for further analysis.

\subsection{Building cluster covariance models}

To build covariance models which captured the diversity of sequences represented by each cluster, I searched the ten CMs with the highest sum of RSS scores in each cluster against the set of genomes which contributed motifs to the cluster. Sequences on which at least four CMs agreed on with an E-value of $<$ 1 were collected. The redundancy of the collected sequences was iteratively reduced in an alignment-free fashion using cd-hit \parencite{Li2006} with the parameters -G 0 -aL 0.1 -aS 0.3 until there were less than 2000 sequences remaining or there were no remaining sequences with $>$ 85\% nucleotide identity. Sequences were extended by 20 bases on each side to capture features which may not have been in the CMfinder-derived motifs, e.g. poorly conserved poly-U tracts. The resulting set of sequences was aligned using MAFFT Q-INS-i \parencite{Katoh2008} using McCaskill base-pairing probabilities \parencite{McCaskill1990}, and secondary structures were predicted using CentroidAlifold \parencite{Hamada2009}, again with McCaskill base-pairing probabilities. CMs were built from the resulting cluster alignments, and sequences which did not match the CM with a bitscore of at least 20 were iteratively discarded. The resulting alignments were then manually curated using RALEE \parencite{Griffiths-Jones2005}, trimming non-conserved flanking sequence and extending the predicted secondary structure where possible. Conserved stop codons were specifically trimmed, so as not to bias subsequent searches.

\subsection{Genome annotation}

The resulting cluster CMs were searched over the initial 1853 embl files. Bitscore thresholds were set for hit significance for each cluster CM using shuffled sequence. Specifically, each cluster model was also used to search a dinucleotide shuffled database of these same 1853 embl file. For each model, a Gumbel distribution was fitted to the distribution of bitscores over the shuffled database, and this null Gumbel distribution was used to compute P-values for hit significance in the native sequences. P-values were corrected for multiple hypothesis testing using the method of \textcite{Benjamini1995}.

\subsection{Analysis of expression data}

\section{Results}

\subsection{Kingdom-wide motif discovery}

The pipeline I developed for discovering putative termination motifs consisted of 3 major stages (see figure X): genome-wise motif discovery with CMfinder  \parencite{Yao2006}, clustering of motifs using a novel similarity measure and the MCL algorithm \parencite{Enright2002}, and manual curation of the resulting motif clusters.  

In the first stage I extracted sequence from -20 to +80 with respect to annotated stop sites, which were then filtered on predicted structural potential to screen for sequences with stronger structures than predicted by their dinucleotide content alone (see Methods). For each genome, I used the resulting set of sequences as input for the CMfinder alogrithm \parencite{Yao2006}. Briefly, CMfinder uses heuristic sequence search, thermodynamic and mutual information-based predictions of secondary structure, and CM-based searches within an EM framework to iteratively discover and refine potential structured RNA motifs, returning a multiple sequence alignment and corresponding CM. CMfinder has previously been successfully used as part of pipelines for the discovery of non-coding RNAs in bacteria \parencite{Weinberg2007, Weinberg2010} and eukaryotes \parencite{Torarinsson2008a}, as well as in our previous discovery of the TRIT element \parencite{Gardner2011a}. Applying this algorithm to the filtered sequences for each genome resulted in a total of 22310 motif predictions. I searched these CMs back over the genome they were predicted from and removed from consideration motifs with very low ($<$100) or very high ($>3000$) copy number, or were not enriched with respect to gene terminal regions, leaving a set of 4359 putative termination motifs, approximately 2.5 per organism.

To reduce the complexity of this data set, I developed a method for clustering CMs. Two previous approaches to comparing CMs have been described in the literature. The first, known as CMcompare \parencite{Honer-zu-Siederdissen2010}, computes the score of a so-called `link sequence', that is a sequence with the highest value of $\min{(S_1(s), S_2(s))}$, where $S_x(s)$ is the score of sequence $s$ with respect to model $S_x$. While this has been proposed as a measure of CM specificity in the context of the Rfam database, it is unclear how accurately this single link sequence captures the overlap between the sequence spaces described by two CMs, let alone the reality of overlaps in actual biological sequence databases. A second method, proposed as part of the Evofam pipeline for automated ncRNA family discovery in eukaryotic genome alignments \parencite{Parker2011}, approximates the Kullback-Leibler divergence between two CMs, that is the similarity of the probability distributions over sequences emitted by the two CMs, using the difference in Infernal CM E-value calculations on a human reference sequence from each model's training set. In the context of the Evofam pipeline, the use of the human reference sequence is justifiable, as the study was primarily concerned with the discovery of ncRNA families present in the human genome. However, in the present case of clustering motifs across an entire domain of life, there is no obvious single sequence to use as a reference for the purposes of a comparison between every pair of CMs.

\subsection{Lineage-specific differences in RIT utilization}

\subsection{Non-canonical putative attenuators of transcription}

\section{Discussion}